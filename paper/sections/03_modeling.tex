\section{系统建模}\label{sec:model}

\textbf{核心科学问题。} 如何设计、建模和优化一个可扩展的分层RF--RIS--MRC无线输能系统,使其能够在复杂的非视距环境中可靠地传输能量,并高效地为大规模WSN部署供能,同时在统一框架内协调长距离RF能量路由和近场磁共振充能?该问题自然分解为以下三个紧密耦合的子问题:

\textbf{子问题1:多RIS协作NLOS环境能量路由。} 如何在直接视距链路不存在的复杂地形中构建和优化多跳RIS辅助RF能量路径?具体包括:(i) 如何建模真实地形(如山地、障碍物)上的多RIS级联反射信道,并将地形高程和阴影纳入能量路由?(ii) 如何联合设计能量路由路径和RIS相位配置,以最大化接收功率、最小化路径损耗或平衡多簇能量分配?(iii) 如何保持对环境变化(如植被、部分遮挡)的鲁棒性,并支持扩展到额外的RIS面板和WSN簇?

\textbf{子问题2:RIS增强MRC近场能量聚焦与鲁棒性。} 如何将可编程超表面用作近场能量透镜,以增强簇内磁共振耦合效率和错位容忍度?具体包括:(i) RIS结构重塑近场磁场分布并改善耦合系数、品质因数和端到端效率的底层机制是什么?(ii) RIS结构参数(单元几何、周期性、材料特性)如何影响聚焦强度、效率和线圈位移/旋转鲁棒性之间的权衡?(iii) 如何设计自适应RIS控制模式,以补偿实际部署中传感器节点的空间漂移或位置不确定性?

\textbf{子问题3:RF--RIS--MRC--WSN分层系统的跨层协调。} 如何构建统一的、可扩展的能量流框架,协调长距离RF能量路由、簇级MRC充能和WSN工作负载需求?具体包括:(i) 如何将分层RF--RIS--MRC--WSN系统建模为能量流图,并在每层定义约束(功率预算、RIS配置资源、硬件限制)?(ii) 如何设计跨层优化算法,联合分配RF和MRC层之间的功率,选择RIS能量路由,并根据流量负载和剩余能量调度簇级功率分配?(iii) 如何确保所得架构在网络规模方面保持可扩展性,并可扩展到新组件(如无人机载RIS、移动机器人、6G物联网节点),而无需重新设计整个系统?

\textbf{网络与能量模型。} 多个簇部署在遮挡地形中;簇头(CH)具备 RF 接收与 MRC 发射,簇内节点具备 MRC 接收。多块 RIS 形成可控反射多跳,将 RF 功率从源端路由到各 CH;路径损耗、遮挡和 RIS 相移决定接收功率;近场耦合系数刻画 MRC 效率与偏移敏感度。

\textbf{信息模型(AOEI)。} 各节点/簇头维护能量状态信息(ESI),能量信息年龄 AOEI 衡量新鲜度,超过阈值触发优先上报以指导能量路由与充能决策。

\textbf{控制时序。} 帧划分为规划与执行时隙:ALDP 规划 RF 多跳与 MRC 充能时长;EETOR 选择携带 ESI 的机会式能量路由。

\textbf{指标。} (i) 寿命:首节点失效时间;(ii) 能量均衡:剩余能量变异系数;(iii) 信息新鲜度:平均/尾部 AOEI;(iv) 输能效率:接收 RF/MRC 功率与端到端效率。
