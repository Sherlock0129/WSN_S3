\section{系统建模}\label{sec:model}

本章从“节点建模”和“场景建模”两个层面给出我们所建立的全部模型,覆盖能量、近场耦合/超表面、计算通信负载、存储与控制开销等,为后续机制设计与优化奠定基础。

\subsection{节点建模}\label{subsec:node_model}
本节针对四类节点——传感器节点(Sensor)、可重构智能反射面(RIS)、RF 簇头(Cluster Head, CH)与 Sink(信息汇聚与调度)——分别给出能量、耦合/物理、计算通信与存储控制等模型。

\subsubsection{传感器节点(Sensor)}
\paragraph{(a) 能量模型:光伏获能 + 用能 + 向 CH 的 MRC 送能}
传感器节点通过太阳能获能,并以近场 MRC 方式向 CH 反向送能。设每帧周期为 \(T_f\)。
- 光伏获能:
\begin{equation}
\begin{aligned}
P_{\text{PV}}(t) &= \eta_{\text{pv}}\,A_{\text{pv}}\,G_t(t)\,\Xi(\psi,\theta_{\odot},t),\\
E_{\text{PV}} &= \int_{t}^{t+T_f} P_{\text{PV}}(\tau)\,\mathrm{d}\tau
\end{aligned}
\end{equation}
其中 \(\eta_{\text{pv}}\) 为组件效率,\(A_{\text{pv}}\) 为面积,\(G_t\) 为到达水平面的总辐照度,\(\Xi\in[0,1]\) 为入射角/遮挡修正项(含地形/建筑遮阴)。
- 本地用能:
\begin{equation}
\begin{aligned}
E_{\text{use}} &= P_{\text{sense}}t_{\text{sense}} + P_{\text{comp}}t_{\text{comp}} + P_{\text{tx}}^{\text{RF}}t_{\text{tx}}\\
&\quad + P_{\text{rx}}t_{\text{rx}} + P_{\text{idle}}t_{\text{idle}}
\end{aligned}
\end{equation}
- 向 CH 的 MRC 送能:节点以线圈驱动功率 \(P_{u\to k}^{\text{tx}}\) 在时长 \(t_{u\to k}\) 内向所属簇 \(k\) 的 CH 送能,其到达 CH 的能量为
\begin{equation}
\begin{aligned}
E_{u\to k}^{\text{MRC}} &= \eta_{u\to k}^{\text{MRC}}\, P_{u\to k}^{\text{tx}}\, t_{u\to k},\\
0 &\le \eta_{u\to k}^{\text{MRC}}\le 1
\end{aligned}
\end{equation}
其中效率 \(\eta_{u\to k}^{\text{MRC}}\) 由耦合系数、品质因数与可能的 RIS 聚焦共同决定(见下)。节点侧驱动损耗由逆变/功放效率 \(\eta_{\text{drv}}\) 计入:\(E_{\text{drv}}=P_{u\to k}^{\text{tx}}t_{u\to k}/\eta_{\text{drv}}\)。
- 电池演化:
\begin{equation}
E_{u}(t+T_f)=\min\{E_{\max},\; E_u(t)+E_{\text{PV}}-E_{\text{use}}-E_{\text{drv}}\}.
\end{equation}

\paragraph{(b) 近场耦合/物理模型}
两线圈谐振耦合效率近似为
\begin{equation}
\begin{aligned}
\eta_{u\to k}^{\text{MRC}} &\approx \frac{k_{u,k}^2 Q_u Q_k}{\big(1+\sqrt{1+k_{u,k}^2 Q_u Q_k}\big)^2},\\
k_{u,k} &= \frac{M_{u,k}}{\sqrt{L_u L_k}}
\end{aligned}
\end{equation}
互感 \(M_{u,k}=M(d,\Delta z,\theta)\) 随水平/垂直偏移与姿态角变化。若存在近场 RIS/超表面,其增益以函数 \(\mathcal{F}_{\text{RIS}}\) 修正:\(\eta\leftarrow \eta\cdot\mathcal{F}_{\text{RIS}}(\text{结构, 频率, 姿态})\)。

\paragraph{(c) 计算与通信负载模型}
- 业务到达:每帧产生数据 \(A_u(t)\)(可泊松/伯努利事件驱动)。
- 通信:低功耗 RF 向 CH 上传,单位比特能耗 \(\epsilon_{\text{tx}}\)、接收能耗 \(\epsilon_{\text{rx}}\)。
- 计算:本地处理量 \(C_u\)(CPU 周期)与动态功耗 \(P\propto f^\gamma\)。

\paragraph{(d) 存储与控制开销}
- 电池容量 \(E_{\max}\)、荷电状态(SOC)与老化可通过库仑计或衰减模型修正。
- 控制开销:接收 Sink/CH 的调度指令,开销合入 \(E_{\text{use}}\)。

\subsubsection{可重构智能反射面(RIS)}
\paragraph{(a) 反射/超表面模型}
RIS 由 \(N\) 个单元组成,等效反射矩阵
\begin{equation}
\begin{aligned}
\boldsymbol{\Theta} &= \operatorname{diag}(\beta_1 e^{j\theta_1},\ldots,\beta_N e^{j\theta_N}),\\
\theta_n &\in \mathcal{Q},\ \ 0\le\beta_n\le 1.
\end{aligned}
\end{equation}
在近场工作区,RIS 通过改变等效边界条件重塑磁场分布,提升 \(k_{u,k}\) 或有效 \(Q\),其对 \(\eta^{\text{MRC}}\) 的作用以经验/仿真查表函数 \(\mathcal{F}_{\text{RIS}}\) 体现。

\paragraph{(b) 硬件与约束}
相位量化比特 \(b\)、更新速率、插入损耗与功耗 \(P_{\text{RIS}}\)(通常极低)。配置比特数计入控制开销。

\subsubsection{RF 簇头(CH)}
\paragraph{(a) 能量模型:仅由簇成员的 MRC 供能}
CH 的能量来源完全来自本簇成员通过 MRC 反向送能:
\begin{equation}
E_{\text{in}}^{\text{MRC}}=\sum_{u\in\mathcal{U}_k} \eta_{u\to k}^{\text{MRC}}\, P_{u\to k}^{\text{tx}}\, t_{u\to k},\quad E_{\text{CH}}(t+T_f)=E_{\text{CH}}(t)+E_{\text{in}}^{\text{MRC}}-E_{\text{use}}^{\text{CH}}.
\end{equation}
其中 CH 的用能包含:
\begin{equation}
E_{\text{use}}^{\text{CH}}=P_{\text{rx}}^{\text{RF}}t_{\text{rx}}+P_{\text{tx}}^{\text{RF}}t_{\text{tx}}+P_{\text{comp}}t_{\text{comp}}+P_{\text{ctrl}}t_{\text{ctrl}}+P_{\text{idle}}t_{\text{idle}}.
\end{equation}
\paragraph{(b) 通信与计算模型}
CH 负责聚合与上行到 Sink 的 RF 数据通信(不向 Sink 传能),链路速率 \(R_{k\to S}\) 与功耗遵循所选制式的能耗模型;本地聚合/压缩计算量 \(C_k\) 由业务决定。
\paragraph{(c) 存储与队列}
维护数据队列 \(Q_k\):\(Q_k(t+1)=\max\{Q_k(t)-\mu_k(t),0\}+\sum_{u}A_u(t)\),其中服务率 \(\mu_k\) 由 CH\(\to\)Sink 的可用时隙与信道条件决定。

\subsubsection{Sink(信息汇聚与调度)}
Sink 负责控制与数据汇聚,不参与能量传输。其功能与模型包括:
- 控制平面:收集能量/业务状态摘要,运行时隙/功率/RIS 配置的调度算法,并下发给 CH/RIS/节点;控制能耗计入各受控节点,不计入能量链路。
- 业务层:汇聚各 CH 上行数据,存储与转发开销可忽略或以固定功耗近似(根据实现)。

% ===================== 场景建模 =====================
\subsection{场景建模}\label{subsec:scene_model}

\subsubsection{全局场景建模:Carrauntoohil 山地数据驱动}\label{sec:carrauntoohil}
我们选取爱尔兰最高峰 Carrauntoohil 山地作为目标场景建模对象。该区域具有典型性:冬季太阳高度角低,山阴面(背阳坡)长时间处于阴影,导致光伏可用性强烈非均匀,这对“节点\(\to\)CH 的 MRC 供能”和“CH\(\to\)Sink 的上行通信”形成耦合约束。

\paragraph{区域选取与数据来源。} 在卫星底图上框选研究区域,由于平台无法直接导出规则网格的高程 DEM,我们采用“稀疏采样 + 插值复原”的方法:
- 在区域内进行 \(N\) 个高程采样(经纬度与地面高程),尽量覆盖山脊、谷地、湖泊与坡向变化显著处;
- 采样点记为 \((\lambda_i,\varphi_i,h_i)\), \(i=1,\ldots,N\)。

\paragraph{坐标与投影。} 以采样中心或首点为原点 \(O\)(经纬高 \(\lambda_0,\varphi_0,h_0\)),将 WGS-84 坐标映射到局部 ENU 米系:
\begin{equation}
\begin{aligned}
x&=(\lambda-\lambda_0)\cos\varphi_0\,R_\oplus,\\
y&=(\varphi-\varphi_0)\,R_\oplus,\\
z&=h-h_0,\quad R_\oplus\approx 6371\,\text{km}.
\end{aligned}
\end{equation}
\begin{figure*}[!t]
  \centering
  \includegraphics[width=.95\linewidth]{sections/figures/elev_coord.png}
  \caption{高程建系与场景边界示意。彩色底图为插值后的高程场 \(H(x,y)\),坐标系为局部 ENU;比例尺与轴向标注对应 \((x,y)\) 的米尺度。该图对应 \(H:\Omega\to\mathbb{R}_{+}\) 的连续近似与研究域 \(\Omega\) 的选择。}
  \label{fig:elev_coord}
\end{figure*}

\paragraph{地形复原:三角网插值 + 栅格化。} 构建地形:
- Delaunay 三角剖分得到 TIN:\(\mathcal{T}=\operatorname{Del}({(x_i,y_i)}_{i=1}^{N})\);
- 在每个三角形内以重心/仿射插值恢复 \(z(x,y)\) 的连续近似(形成分段平面曲面);
- 将 \(z(x,y)\) 以分辨率 \(\Delta\) 栅格化为高度场 \(H[m,n]\),得到规则 DEM(便于快速可见性/阴影运算)。
\begin{figure}[!t]
  \centering
  \includegraphics[width=.95\linewidth]{sections/figures/sampling_tin.png}
  \caption{稀疏采样与描点建模:TIN插值与规则DEM栅格化示意。}
  \label{fig:sampling_tin}
\end{figure}

% —— 新增:采样点分布与三维重建可视化 ——
\begin{figure*}[!t]
  \centering
  \includegraphics[width=.8\linewidth]{sections/figures/points_group.png}
  \caption{采样点的三维分布与分组示意。彩色散点表示不同位置处的高程/分组,可用于说明稀疏采样的空间覆盖与结构特征。}
  \label{fig:points_group}
\end{figure*}

\begin{figure*}[!t]
  \centering
  \includegraphics[width=.8\linewidth]{sections/figures/rebuild_3D.png}
  \caption{基于稀疏样本的地形三维复原结果。通过 TIN/插值获得的连续曲面,展示地形起伏与坡向。}
  \label{fig:rebuild_3D}
\end{figure*}

\paragraph{阴影与辐照修正。} 基于复原 DEM 计算典型日期的太阳位置 \((\alpha(t),\,\gamma(t))\)(高度角/方位角)。对每个栅格中心 \((x,y)\):
- 计算坡度/坡向 \((s,\,\phi)\) 与法向量 \(\mathbf{n}(x,y)\);
- 沿太阳入射方向进行地形遮挡投射,得到遮阴指示 \(S(x,y,t)\in\{0,1\}\);
- 形成光照修正因子 \(\Xi(x,y,t)=S(x,y,t)\cdot\max\{0,\mathbf{n}\cdot\mathbf{s}(t)\}\)。
\begin{figure}[!t]
  \centering
  \includegraphics[width=.95\linewidth]{sections/figures/hillshade_back.png}
  \caption{山阴(背阳坡)与地形自遮挡。给定太阳方向单位向量 \(\mathbf{s}(t)\),对每个 \((x,y)\) 取地形法向 \(\mathbf{n}(x,y)\) 与地平角约束,得到遮阴指示 \(S(x,y,t)\in\{0,1\}\)。若沿 \(\mathbf{s}(t)\) 的射线首次与曲面 \(z=H(x,y)\) 相交于当前点之前,则 \(S=0\)。}
  \label{fig:hillshade_back}
\end{figure}
\begin{figure}[!t]
  \centering
  \includegraphics[width=.95\linewidth]{sections/figures/shadow_model.png}
  \caption{阴影建模与照度修正。瞬时有效入射 \(I(x,y,t)=G_0(t)\,\max\{0,\mathbf{n}(x,y)\!\cdot\!\mathbf{s}(t)\}\,S(x,y,t)\),其中 \(G_0(t)\) 为无地形时的水平面辐照度。将 \(I\) 归一化得到 \(\Xi(x,y,t)=I/G_0\),并用于 PV 获能模型。}
  \label{fig:shadow_model}
\end{figure}
\begin{figure}[!t]
  \centering
  \includegraphics[width=.95\linewidth]{sections/figures/solar_geom_aspect.png}
  \caption{太阳几何与坡向入射差异。上:某一日的太阳高度角随小时变化;下:南坡与北坡(坡度相同、坡向相差约180°)的相对入射(与入射角余弦成正比)理论曲线。冬季高纬度下,南坡全天相对入射显著大于北坡,解释了背阳坡光伏不足。}
  \label{fig:solar_aspect}
\end{figure}

\paragraph{可见性与上行数据链路损耗(仅数据,不传能)。} 在 DEM 上对 CH\(\to\)Sink 的 RF 数据上行进行 LoS 判定与路径损耗参数选择:
- 在两点连线方向对 \(H[m,n]\) 逐步采样,若 \(z_{\text{ray}}(d)\le H(d)\) 则标记为 NLoS;
- LoS/NLoS 决定不同路径损耗指数/阴影项;
- 注:本工作不设 RF 能量源,RIS 仅用于簇内近场磁场聚焦,不参与远场路由。

\paragraph{簇与锚点。} 在湖泊及周边地形语义的引导下布设若干簇与 CH,并在 CH 周边选取一至多处 RIS 候选位以进行近场聚焦对比。Sink 位于山谷低处以模拟背阳通信与汇聚节点场景。图~\ref{fig:anchor_dist} 展示典型锚点间的几何距离,用于“走廊宽度选择”“可见性步长”和“路径损耗自变量”的量化。
\begin{figure*}[!t]
  \centering
  \includegraphics[width=.95\linewidth]{sections/figures/satellite_distance.png}
  \caption{关键锚点与几何距离示意。连线长度标注为 \(\lVert \mathbf{a}-\mathbf{b}\rVert\),用于:(1) 估算走廊半宽 \(W\) 和搜索域 \(\mathcal{C}\);(2) 作为路径损耗与时延的几何自变量;(3) 用于可见性判定的步长约束。}
  \label{fig:anchor_dist}
\end{figure*}

\paragraph{数据产物与可复现性。} 该流程生成:
- 稀疏 TIN 与规则 DEM 栅格(\(H[m,n]\));
- 冬季时段的遮阴时间序列 \(S(x,y,t)\) 与光照修正 \(\Xi(x,y,t)\);
- CH\(\leftrightarrow\)Sink 的 LoS 图与损耗参数场;
- ENU 坐标下的节点/簇/湖泊矢量底图。
上述产物直接驱动仿真与可视化,并与第~\ref{subsec:node_model} 节的 PV 与上行链路模型无缝对接。

\subsubsection{簇内场景建模:近场 MRC + RIS 聚焦}\label{sec:cluster_scene}
\paragraph{几何与错位。} 对簇 \(k\) 内任意节点 \(u\) 与 CH 的几何关系用 \(d_{k,u},\,\Delta z_{k,u},\,\theta_{k,u}\) 表征,互感 \(M_{k,u}=M(d_{k,u},\Delta z_{k,u},\theta_{k,u})\),耦合系数 \(k_{k,u}=M_{k,u}/\sqrt{L_u L_k}\),效率 \(\eta_{k,u}^{\text{MRC}}\) 由前述公式给出。我们将“错位”视作 \(\boldsymbol{\delta}_{k,u}=(\delta x,\delta y,\delta z,\delta\theta)\) 的随机扰动,其分布由部署与环境振动决定。

\paragraph{RIS 对远距与错位的补偿。} 在 CH 周边布置近场 RIS/超表面,作为“能量透镜”提升远距离(较大 \(d_{k,u}\))节点的有效 \(k_{k,u}\) 并降低 \(\eta_{k,u}^{\text{MRC}}\) 对 \(\boldsymbol{\delta}_{k,u}\) 的敏感度。我们以查表/仿真函数 \(\mathcal{F}_{\text{RIS}}(d,\Delta z,\theta;\,\text{结构},\text{频率})\) 对效率进行修正:
\begin{equation}
\eta_{k,u}^{\text{MRC}}\;\leftarrow\;\eta_{k,u}^{\text{MRC}}\cdot \mathcal{F}_{\text{RIS}}(d_{k,u},\Delta z_{k,u},\theta_{k,u}),\quad \partial\eta/\partial\boldsymbol{\delta}\ \text{降低}.
\end{equation}
其中“降低偏导”表示错位鲁棒性的提升。

\paragraph{图示。} 图~\ref{fig:mrc_cluster} 展示了簇内场景:RIS 位于 CH 附近,通过重塑近场磁场,使得远处或偏移/倾斜的节点在 RIS 的聚焦下仍能实现较高的输能效率。该图对应局部层(cluster-level)的建模与设计。
\begin{figure*}[!t]
  \centering
  \includegraphics[width=.95\linewidth]{sections/figures/mrc.png}
  \vspace{-0.4em}
  \caption{簇内场景建模示意:RIS 增强远处节点的 MRC 传能效率,并降低线圈不对齐(平移/倾斜)造成的效率损失。}
  \label{fig:mrc_cluster}
\end{figure*}

\paragraph{调度与约束。} 在帧 \(T_f\) 内,簇内送能时隙集合 \(\{t_{u\to k}\}\) 需满足 \(\sum_{u\in\mathcal{U}_k} t_{u\to k} \le T_f- t_{\text{comm}}-t_{\text{comp}}\)。在 RIS 配置比特与更新速率受限下,选择有限个模式实现对“远距优先/错位补偿”的策略化支持。

\paragraph{小结。} 本节将场景建模拆分为全局(地形、光照、可见性)与簇内(近场几何、RIS 聚焦、调度),并在簇内侧引入基于 RIS 的远距增强与错位鲁棒性建模。