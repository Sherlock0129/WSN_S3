\section{实验设计与结果}\label{sec:experiments}
本节围绕第~\ref{sec:model} 节的系统/场景建模与第~\ref{sec:mechanism} 节的机制设计,给出一致、可复现的实验方案与对比结果。我们遵循“全局—局部—跨层”三层结构设计实验:
- 全局层(Global):在真实地形与LoS/NLoS约束下,评估多RIS协作的几何机会路径(第~\ref{subsec:ris_placement_mech} 节)的能量可达性与增益;
- 局部层(Cluster):在簇内几何与错位扰动下,量化RIS能量透镜对MRC效率与鲁棒性的提升(第~\ref{sec:cluster_scene} 节);
- 跨层协调(X-Layer):在统一能量流图中联合调度帧级时隙/功率分配与RIS配置,评估端到端供能能力与网络寿命(第~\ref{sec:mechanism} 节)。

\subsection{实验问题与假设}
我们聚焦四个核心问题:
- Q1(全局可达性):在背阴/遮挡地形中,多RIS多跳机会路径能否显著抬升“源—CH”的接收功率与能量可达率?
- Q2(局部鲁棒性):RIS近场能量透镜能否扩大“有效补能体积”,降低MRC对线圈错位/倾斜的敏感度?
- Q3(端到端协同):当全局(多RIS)与局部(RIS+MRC)同时启用时,跨层协调能否降低能量短缺率并延长网络寿命?
- Q4(代价与可扩展):在RIS数量、相位量化比特与控制更新速率受限下,可达增益与控制开销/复杂度之间的折中如何?

\subsection{场景、数据与实现}
\paragraph{地形与几何(Carrauntoohil)。} 采用第~\ref{sec:carrauntoohil} 节的数据驱动建模:稀疏采样—TIN—栅格DEM流程获得$H[m,n]$,并在冬季日轨迹下计算遮阴$S(x,y,t)$与光照修正$\Xi(x,y,t)$。CH与sink、RIS候选及走廊由图~\ref{fig:ris_distribution_math} 所示的机制生成。

\paragraph{两类实验配置。}
- 配置A(与第3章严格一致):不设RF能量源;CH\,$\to$\,sink仅进行数据上行;RIS仅用于簇内近场聚焦MRC(验证Q2与局部贡献)。
- 配置B(第4章扩展验证):在向阳坡增设RF能量源,按第~\ref{subsec:ris_placement_mech} 节构建多RIS反射链路,将能量路由至背阴簇头(验证Q1/Q3/Q4)。两配置共享同一地形/业务/能量参数,以隔离变量。

\paragraph{工作负载与时间轴。} 采用典型WSN占空:每帧$T_f$内,传感器周期性采样与簇内上报(每小时),CH聚合并按固定日程回传至sink(每日)。跨层协调在帧级进行,RIS配置在日级/小时级少量更新,控制开销计入能耗。

\paragraph{实现与参数来源。} 能量、通信与近场耦合采用第~\ref{sec:model} 节的模型;几何候选、LoS与多跳增益评估采用第~\ref{sec:mechanism} 的公式(含走廊、视域与多跳链路)。参数来自:组件效率$\eta_{\text{pv}}$、面积$A_{\text{pv}}$、MRC品质因数$Q$与耦合系数$k$模型、RIS单元规模与量化比特$b$;地形、遮阴与太阳几何由DEM与天文模型给出。默认参数及敏感性范围在附表给出(频段、安装高、走廊半宽、候选密度、$b\in\{1,2,3\}$ 等)。

\subsection{对比方法与消融设置}
为与三层机制对齐,我们将方法组件化,并进行组合与消融:
- G:全局多RIS几何机会路径(单/双/多跳,$m\le5$),含走廊—视域约束与束搜索裁剪;
- L:簇内RIS近场能量透镜(查表/仿真增益$\mathcal{F}_{\text{RIS}}$)提升MRC效率并补偿错位;
- C:跨层能量流协调(帧级时隙/功率/RIS模式联合调度)。

\noindent 组合体制:
- Baseline-0:无RIS,传统簇内MRC + 既有机会路由/调度(不含跨层);
- Baseline-1(L-only):仅启用簇内RIS透镜;
- Baseline-2(G-only):仅启用全局多RIS路径(配置B);
- Ours-GL:G+L,未启用跨层协调;
- Ours-GLC(完整):G+L+C,端到端联合。

\noindent 关键消融:
- 跳数与布局:$m\in\{1,2,3,4,5\}$,走廊半宽$W$、候选密度、RIS安装高$h_{\text{RIS}}$;
- 量化与硬件:相位量化$b\in\{1,2,3\}$、插入损耗、面板规模$N$;
- 错位鲁棒性:平移$\delta r$、倾角$\delta\theta$的分布强度(小/中/大扰动);
- 控制与反馈:RIS更新周期(小时/日)、状态摘要开销、反馈丢包率。

\subsection{评价指标}
- 寿命:首节点失效时间(FND)、50%节点失效时间(HND);
- 能量短缺:帧级不可达率/短缺率(缺电导致任务跳过);
- 能量均衡:SOC变异系数(CV)与CH能量缓冲区溢出/枯竭次数;
- 端到端效率:源\,$\to$\,CH接收功率/能量与效率(G启用时),节点\,$\to$\,CH的MRC效率(L启用时);
- 任务完成率:数据上报/回传达成率(与能量状态耦合);
- 控制与计算开销:RIS配置比特、调度指令比特与边缘计算能耗;
- 复杂度:候选规模、束搜索宽度与求解时间(离线/在线)。

\subsection{实验流程与统计}
- 初始化:基于DEM生成走廊与候选集,计算视域与LoS图;根据扰动强度采样簇内几何错位;
- 逐日仿真:按日轨迹推进,记录$\Xi(x,y,t)$,执行C(若启用)给出的帧级调度与RIS更新;
- 能量流:在G启用时,按$\pi(t)$选择的多RIS链路路由能量;在L启用时,按$\mathcal{F}_{\text{RIS}}$修正MRC效率;
- 统计:跨不同随机种子与天气型(日照档位)重复,报告均值、标准差与95%置信区间。

\subsection{结果概览}
% 图~\ref{fig:exp_global} 展示了在Carrauntoohil地形上的多RIS几何连通示例与接收功率热力图;图~\ref{fig:exp_local} 展示了簇内RIS透镜对MRC效率—错位曲线的抬升与展宽;图~\ref{fig:exp_end2end} 汇总端到端能量流与寿命指标的对比。

% \begin{figure}[!t]
%   \centering
%   \includegraphics[width=.98\linewidth]{ris_rf_sink_map.png}
%   \vspace{-0.5em}
%   \caption{全局层多RIS几何连通与机会路径示例:在走廊与视域约束内,单/双/多跳链路显著抬升背阴侧可达接收功率,提升能量可达率。}
%   \label{fig:exp_global}
% \end{figure}

% \begin{figure}[!t]
%   \centering
%   \includegraphics[width=.86\linewidth]{mrc.png}
%   \vspace{-0.5em}
%   \caption{簇内RIS能量透镜示意与效率趋势:在平移/倾斜错位下,RIS透镜提升MRC效率并扩大“有效补能体积”(等效率轮廓外扩)。}
%   \label{fig:exp_local}
% \end{figure}

% \begin{figure}[!t]
%   \centering
%   \includegraphics[width=.98\linewidth]{simulation_energy_results.png}
%   \vspace{-0.5em}
%   \caption{端到端指标对比(示例):与Baseline-0/1/2相比,Ours-GLC在能量短缺率、SOC均衡与寿命(FND/HND)上取得一致改进;控制比特与计算额外开销保持在可接受范围内。}
%   \label{fig:exp_end2end}
% \end{figure}

\begin{figure}[!t]
  \centering
  \includegraphics[width=.98\linewidth]{simulation_energy_results9false.png}
  \vspace{-0.5em}
  \caption{节点能量随时间变化(单次运行示例)。图示若干节点的SOC/能量轨迹,验证端到端调度与供能的稳定性。}
  \label{fig:exp_energy_time}
\end{figure}

\paragraph{主要发现。}
- 全局(G):在NLoS越岭场景,多RIS链路相对“无RIS直达”显著抬升CH侧接收功率,双/三跳对复杂地形尤为有效;安装高与走廊半宽存在最佳区间。
- 局部(L):RIS透镜对中/大错位下的MRC效率提升显著,等效“补能体积”扩大,降低了对机械精度与定位误差的敏感度。
- 跨层(C):在G+L基础上加入帧级联合调度后,能量短缺率进一步下降,SOC分布更均衡,首节点与中位数寿命提升;增益对相位量化与更新速率不敏感($b\ge2$足够)。
- 代价与扩展:在相同控制预算下,优先增加RIS数量与安装高往往比提高量化比特更“划算”;方法对新增簇/面板与UAV-RIS具备可扩展性(见附录扩展实验)。

\paragraph{一致性与可复现性。} 配置A用于与第3章假设完全对齐(无RF能量源,RIS仅簇内);配置B验证第4章提出的多RIS能量路由扩展。两者共享同一实现与统计流程,避免混杂变量。完整参数、随机种子与脚本入口在补充材料与仓库附表中给出,可一键复现实验与绘图。
