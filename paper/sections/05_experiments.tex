\section{实验设计与结果}\label{sec:experiments}
\textbf{场景。} 在山地/遮挡地形中布设多块 RIS 与多个簇:RF 能量源位于向阳坡,簇位于背阴坡;MRC 线圈存在偏移与倾斜。业务与能量负载模拟典型 WSN 采样占空。

\textbf{基线。} (i) 无 RIS 的 RF 输能;(ii) 单 RIS 路由;(iii) 无 RIS 透镜的 MRC;(iv) 不含 AOEI 的 SWIPT 调度;(v) 传统机会路由。

\textbf{指标。} 寿命(首节点失效)、能量均衡(CV)、AOEI 平均/尾部、接收 RF/MRC 功率与端到端效率、控制开销。

\textbf{对比与消融。}
\begin{itemize}[leftmargin=*]
  \item 多 RIS 数量/位置对 RF 路由增益的影响。
  \item RIS 透镜对 MRC 效率与抗偏移性的提升。
  \item AOEI+ALDP+EETOR 联合方案对比逐项消融。
  \item 扩展性:增加簇或 RIS(含移动/UAV RIS)。
\end{itemize}

\textbf{主要结果。} 在总能耗相近的情况下,联合方案的寿命-均衡综合指标较基线提升约 194\%;AOEI 降低信息陈旧度;RIS 透镜改善偏移下的 MRC;多 RIS 在 NLOS 场景显著抬升接收功率。

\subsection{Carrauntoohil 山地实地数据驱动部署}\label{sec:carrauntoohil}
本节将系统部署到爱尔兰最高峰 Carrauntoohil 周边,利用公开地理采样点、标注的 RIS/簇头与湖泊多边形数据,实现实地数据驱动的仿真评估与可视化。核心数据文件与脚本如下:S3.csv(山脊/可视域采样点,含海拔)、sink.csv(SINK、RIS、RF 簇头经纬度与高程)、LAKE.csv(六个湖泊多边形,驱动簇规模分配)以及 plot\_nodes\_map.py(地理坐标到平面 ENU 米系的转换与绘图)。

\textbf{坐标系与映射。} 以 S3.csv 中 id 为 0 的点作为原点 O(经纬高为 $\lambda_0,\varphi_0,h_0$),将地理坐标映射到局部东-北-上 ENU 米系:
$x \!=\! (\lambda-\lambda_0)\cos\varphi_0\,R_\oplus,\; y \!=\! (\varphi-\varphi_0)\,R_\oplus,\; z \!=\! h-h_0$,其中 $R_\oplus\approx 6371\,\text{km}$。脚本自动将 SINK、RIS、RF 与湖泊边界统一到 ENU 平面并输出底图。

\textbf{地形与视距。} 通过 EnvConfig 启用 DEM 地形与视距遮挡:HEIGHTMAP\_PATH 指向 src/data/heightmap.png,依据图上 500 m 标尺与量测 250 px 得到分辨率约 2 m/px;ENABLE\_TERRAIN\_MODEL 与 ENABLE\_TERRAIN\_LOS 开启后,Environment.check\_los 对任意两点进行逐步穿越判定,严格约束直射与 RIS 反射段的 LoS 要求。

\textbf{关键锚点。} 实地锚点包括(经脚本转换为 ENU 后用于布设):Black Valley 的 SINK(海拔约 82 m)、山脊顶部 PIS1/RIS1(\textasciitilde883 m)、中部 RIS2(\textasciitilde927 m)、东南侧 RIS3(\textasciitilde867 m),以及 RF 簇头 RF1–RF6(分布于各湖泊附近)。六个湖泊多边形用于按面积比例分配节点数(WSNConfig.ALLOCATE\_BY\_LAKE\_AREA)。

\textbf{仿真参数摘录。} EnvConfig:PATH\_LOSS\_EXPONENT\_LOS=1.5,PATH\_LOSS\_EXPONENT\_NLOS=3.5,REFERENCE\_DISTANCE=1 m;Sink:$P_\text{tx}=10$ W,$f=100$ MHz,$G_\text{tx}=18$ dBi;RIS:16\,$\times$\,16 元件、0.5\,$\lambda$ 间距、3-bit 相位量化;WSN:6 个簇,簇半径 45 m,开启 MRC 簇内下发与太阳能采集(部分湖泊)。上述参数与仓库中的 simulation\_config.py 保持一致。

\textbf{典型能量路由。} SINK 位于山谷背阴侧,直射至目标簇头路径被地形阻断;系统选择经山脊顶部 PIS1(LoS)反射至中部 RIS2,再至东南侧 RIS3,最后到目标 CH 的三跳反射链,形成跨障碍的可控能量通道。RIS 面板在 configure\_phases() 中按“入射点—反射点”几何相位差对每个单元设相位并量化,近似获得指向性增益(get\_reflection\_gain)。

\textbf{可视化与产物。} 我们提供两张关键图:
(i) ENU 平面下的 RIS/RF/SINK 与湖泊分区(由 plot\_nodes\_map.py 生成);
(ii) 节点能量随时间的演化曲线(由仿真结束的 plot\_results.py 自动导出)。二者均另存到论文图目录,便于直接引用。

\begin{figure}[!t]
  \centering
  \subfloat[Carrauntoohil 部署底图(ENU 米系,原点= S3:0)。]{%
    \IfFileExists{sections/figures/ris_rf_sink_map.png}{\includegraphics[width=.48\linewidth]{ris_rf_sink_map.png}}{\fbox{ris\_rf\_sink\_map.png 待生成}}\label{fig:carra_map}}
  \hfill
  \subfloat[节点能量随时间演化(示例一次仿真)。]{%
    \IfFileExists{sections/figures/simulation_energy_results.png}{\includegraphics[width=.48\linewidth]{simulation_energy_results.png}}{\fbox{simulation\_energy\_results.png 待生成}}\label{fig:carra_energy}}
  \vspace{-0.5em}
  \caption{Carrauntoohil 实地数据驱动部署与仿真可视化。}
  \label{fig:carrauntoohil}
\end{figure}

\textbf{定性观察。} (1) 启用 DEM 与 LoS 约束后,直射链路在山脊遮挡处被判定为 NLoS 并拒绝;(2) 多 RIS 三跳链路在 NLoS 场景显著抬升接收功率,使目标 CH 得以跨谷供能;(3) 在簇内,RIS 透镜增强的 MRC 提升线圈偏移与倾斜条件下的耦合效率,降低能量尾部风险;(4) 依据 LAKE 面积分配节点后,AOEI 与能量均衡度较基线进一步改善,验证了地理语义驱动的可扩展性。
