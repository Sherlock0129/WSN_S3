\section{实验设计与结果}\label{sec:experiments}
\textbf{场景。} 在山地/遮挡地形中布设多块 RIS 与多个簇:RF 能量源位于向阳坡,簇位于背阴坡;MRC 线圈存在偏移与倾斜。业务与能量负载模拟典型 WSN 采样占空。

\textbf{基线。} (i) 无 RIS 的 RF 输能;(ii) 单 RIS 路由;(iii) 无 RIS 透镜的 MRC;(iv) 不含 AOEI 的 SWIPT 调度;(v) 传统机会路由。

\textbf{指标。} 寿命(首节点失效)、能量均衡(CV)、AOEI 平均/尾部、接收 RF/MRC 功率与端到端效率、控制开销。

\textbf{对比与消融。}
\begin{itemize}[leftmargin=*]
  \item 多 RIS 数量/位置对 RF 路由增益的影响。
  \item RIS 透镜对 MRC 效率与抗偏移性的提升。
  \item AOEI+ALDP+EETOR 联合方案对比逐项消融。
  \item 扩展性:增加簇或 RIS(含移动/UAV RIS)。
\end{itemize}

\textbf{主要结果。} 在总能耗相近的情况下,联合方案的寿命-均衡综合指标较基线提升约 194\%;AOEI 降低信息陈旧度;RIS 透镜改善偏移下的 MRC;多 RIS 在 NLOS 场景显著抬升接收功率。
