\section{结论}\label{sec:conclusion}

本文面向复杂地形WSN,提出基于RIS的分层混合无线输能架构,构建RIS + WPT + WSN一体化生态系统。本文的核心贡献体现在三个层级:

\textbf{(主)RIS + WPT + WSN一体化生态系统:}通过系统级整合、长期供能、自组织能力和工程实用路线,实现了从理论建模到硬件原型的完整工程实现路径。统一的RF--RIS--MRC--WSN跨层能量流模型协调端到端能量传输,支持大规模WSN部署和未来智能能源基础设施扩展。

\textbf{(辅1)跨山体能量反射:}通过人工电磁路径实现NLOS能量覆盖,特别解决了阴坡补盲问题,突出了山地环境的特殊性与工程意义。RIS构建的可控反射路径突破了地形障碍限制,实现了从向阳坡到背阴坡的能量传输。

\textbf{(辅2)多RIS协同能量路由系统:}将RIS视为能量节点,构建了能量互联网(Energy Routing Network)工程框架。多跳反射链路和能量路径选择机制实现了远距离能量传输和智能能量调度,为未来能量网络奠定了基础。

通过跨层优化(如 ALDP、EETOR 等),系统将需求感知与 RIS 辅助的 RF 多跳路由、RIS 增强的 MRC 充能紧密耦合,在不增加能耗的前提下显著提升寿命与能量均衡。实验显示相较传统 RF/MRC 基线及消融方案均有显著增益。

未来工作:加强RIS控制与数据安全,探索学习驱动的相位优化,扩展能量互联网框架到更多应用场景,并在移动RIS(如无人机载RIS)实地试验中验证系统的工程实用性。
