\section{机制设计}\label{sec:mechanism}

\textbf{多RIS协同能量路由系统(能量互联网框架)。} 将RIS面板视为能量路由网络中的主动节点,构建类似数据网络的能量互联网(Energy Routing Network)。系统通过以下机制实现多跳反射链路和能量路径选择:

\textbf{(1) 多RIS协作RF能量路由。} 构建基于RIS的多跳反射能量路径,实现跨山体、跨障碍物的长距离能量传输。具体包括:(i) \textbf{地形感知路由图构建:}集成数字高程模型(DEM)数据,构建包含RIS节点、能量源和簇头的路由图,考虑地形遮挡和阴影效应;(ii) \textbf{多跳路径优化:}采用动态规划、图搜索或基于ADMM的联合相位-路径优化算法,选择最优RIS反射序列,最大化接收功率或平衡多簇能量分配;(iii) \textbf{自适应重配置:}支持RIS面板的动态增删和相位重配置,适应环境变化(如植被生长、部分遮挡)和网络扩展需求。

\textbf{(2) RIS增强MRC近场能量透镜。} 利用可编程超表面重塑近场磁场分布,增强磁耦合效率并提升对线圈错位的鲁棒性。具体包括:(i) \textbf{磁场聚焦设计:}通过HFSS/CST电磁仿真,设计RIS单元几何和周期性,实现近场磁场聚焦;(ii) \textbf{错位补偿:}分析耦合系数$k$、品质因数$Q$和效率$\eta$在错位情况下的变化,设计自适应RIS控制模式补偿空间漂移;(iii) \textbf{动态调整:}根据传感器节点位置不确定性,实时调整RIS相位配置,维持高效能量传输。

\textbf{(3) 跨层能量流协调。} 构建统一的RF--RIS--MRC--WSN能量流图,实现端到端能量优化。具体包括:(i) \textbf{能量流图建模:}将系统建模为多层级能量流图,定义每层的能量约束(功率预算、RIS配置资源、硬件限制);(ii) \textbf{联合优化算法:}设计跨层优化算法,联合分配RF和MRC层之间的功率,选择RIS能量路由,并根据流量负载和剩余能量调度簇级功率分配;(iii) \textbf{信息-能量协同:}通过AOEI机制将能量状态信息新鲜度与能量路由决策耦合,实现信息驱动的能量调度。

\textbf{(4) AOEI优先机制。} 动态设定能量信息年龄阈值,使新鲜度不足的节点优先上报,将信息新鲜度与输能紧迫度耦合,并沿活跃能量路径插入更新。

\textbf{(5) 自适应Lyapunov时长规划(ALDP)。} 基于Lyapunov优化规划每时隙RF多跳与MRC充能时长,稳定AOEI与能量队列,同时最大化接收功率。

\textbf{(6) 能量高效传输机会路由(EETOR)。} 选择多跳RIS路径同时输能并携带ESI,权衡路径损耗、反射增益与AOEI紧迫度,支持RIS面板的增删与重配置。

\textbf{协同关系。} 多RIS协作路由构建能量互联网骨干,RIS增强MRC提供簇内高效充能,跨层协调实现端到端优化,AOEI、ALDP、EETOR三机制绑定信息新鲜度与能量路由,共同提升系统寿命、能量均衡和工程实用性。
