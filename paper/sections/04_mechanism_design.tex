\section{机制设计}\label{sec:mechanism}

\textbf{多RIS协同能量路由系统(能量互联网框架)。} 将RIS面板视为能量路由网络中的主动节点,构建类似数据网络的能量互联网(Energy Routing Network)。系统通过以下机制实现多跳反射链路和能量路径选择:

\textbf{(1) 多RIS协作RF能量路由。} 构建基于RIS的多跳反射能量路径,实现跨山体、跨障碍物的长距离能量传输。具体包括:(i) \textbf{地形感知路由图构建:}集成数字高程模型(DEM)数据,构建包含RIS节点、能量源和簇头的路由图,考虑地形遮挡和阴影效应;(ii) \textbf{多跳路径优化:}采用动态规划、图搜索或基于凸近似的联合相位-路径优化思想,选择高增益RIS反射序列,最大化接收功率或平衡多簇能量分配;(iii) \textbf{自适应重配置:}支持RIS面板的动态增删和相位重配置,适应环境变化(如植被生长、部分遮挡)和网络扩展需求。

\textbf{(2) RIS增强MRC近场能量透镜。} 利用可编程超表面重塑近场磁场分布,增强磁耦合效率并提升对线圈错位的鲁棒性。具体包括:(i) \textbf{磁场聚焦设计:}通过电磁仿真,设计RIS单元几何和周期性,实现近场磁场聚焦;(ii) \textbf{错位补偿:}分析耦合系数$k$、品质因数$Q$和效率$\eta$在错位情况下的变化,设计自适应RIS控制模式补偿空间漂移;(iii) \textbf{动态调整:}根据传感器节点位置不确定性,实时调整RIS相位配置,维持高效能量传输。

\textbf{(3) 跨层能量流协调。} 构建统一的RF--RIS--MRC--WSN能量流图,实现端到端能量优化。具体包括:(i) \textbf{能量流图建模:}将系统建模为多层级能量流图,定义每层的能量约束(功率预算、RIS配置资源、硬件限制);(ii) \textbf{联合优化算法:}设计跨层优化算法,联合分配RF和MRC层之间的功率,选择RIS能量路由,并根据流量负载与剩余能量需求感知进行簇级功率分配;(iii) \textbf{反馈与轻量控制:}采用低开销的能量状态摘要与周期性校准,避免过多控制开销。

\textbf{(4) 基于DEM与LoS约束的RIS几何选址与机会路径构建(解决问题一)。} 在不改变既定通信时程的前提下(簇成员每小时上报到CH,CH每日回传至sink),以纯数学形式刻画RIS选址与多跳反射链路优化:定义候选集合、视域与走廊约束,构建1--5跳的级联功率最大化问题,以得到用于每日回传段的稳健几何路径。

% ===================== 数学化的RIS选址与机会路径 =====================
\subsection{数学化的RIS选址与机会路径:固定时程下的跨地形连通}
\label{subsec:ris_placement_mech}

% 图:RIS分布图(数学化选址的落地结果)
\begin{figure*}[t]
  \centering
  \includegraphics[width=0.9\linewidth]{sections/figures/ris_distribution.png}
  \caption{数学化RIS选址的几何结果:在DEM与LoS约束下对若干CH/\,sink对进行单/双RIS与多跳束搜索,得到的部署方案与湖泊地理背景(单位:m)。}
  \label{fig:ris_dist}
\end{figure*}

% 修正浮动环境错误和大括号匹配
% ==================== 数学建模段落继续 ====================

\paragraph{地形、节点与视域}
令地形域为$\Omega\subset\mathbb{R}^2$,高程映射为$H: \Omega\to\mathbb{R}_{+}$。对任一点$(x,y)\in\Omega$,其地表点为$(x,y,H(x,y))$。设源簇头(CH)与目标(另一CH或sink)位置分别为
\begin{equation}
\begin{aligned}
\mathbf{p}_\mathrm{s} &= (x_\mathrm{s},y_\mathrm{s},H(x_\mathrm{s},y_\mathrm{s})+h_\mathrm{s}),\\
\mathbf{p}_\mathrm{d} &= (x_\mathrm{d},y_\mathrm{d},H(x_\mathrm{d},y_\mathrm{d})+h_\mathrm{d})
\end{aligned}
\end{equation}
其中$h_\mathrm{s},h_\mathrm{d}>0$为天线/节点安装高。RIS安装高度为常数$h_\mathrm{RIS}>0$,对$(x,y)\in\Omega$,对应候选安装点为$\mathbf{r}(x,y)=(x,y,H(x,y)+h_\mathrm{RIS})$。

定义视域指示函数$\mathsf{L}(\mathbf{a},\mathbf{b})\in\{0,1\}$:若线段$\overline{\mathbf{a}\mathbf{b}}$在任意参数$\tau\in[0,1]$处满足
\begin{equation}
\begin{aligned}
(1-\tau)\,z_{\mathbf{a}}+\tau\,z_{\mathbf{b}} &> H\!\big((1-\tau)\,x_{\mathbf{a}}+\tau\,x_{\mathbf{b}},\,(1-\tau)\,y_{\mathbf{a}}+\tau\,y_{\mathbf{b}}\big)+\varepsilon,
\end{aligned}
\end{equation}
则$\mathsf{L}(\mathbf{a},\mathbf{b})=1$,否则为$0$;$\varepsilon\ge 0$为净空裕度。源/目的视域集合定义为
\begin{equation}
\begin{aligned}
\mathcal{V}_\mathrm{s} &= \{(x,y)\in\Omega:\ \mathsf{L}(\mathbf{p}_\mathrm{s},\mathbf{r}(x,y))=1\},\\
\mathcal{V}_\mathrm{d} &= \{(x,y)\in\Omega:\ \mathsf{L}(\mathbf{r}(x,y),\mathbf{p}_\mathrm{d})=1\}.
\end{aligned}
\end{equation}

\paragraph{走廊与栅格候选}
令$\Pi(\mathbf{p}_\mathrm{s},\mathbf{p}_\mathrm{d})$为连接源/目的在$xy$平面的线段,走廊半宽为$W>0$。走廊集合定义为
\begin{equation}
\mathcal{C}=\big\{(x,y)\in\Omega:\ \operatorname{dist}\big((x,y),\Pi(\mathbf{p}_\mathrm{s},\mathbf{p}_\mathrm{d})\big)\le W\big\}.
\end{equation}
候选集合可取为$\mathcal{S}=\mathcal{C}\cap\Omega$的均匀采样子集,或直接以$\mathcal{V}_\mathrm{s}\cap\mathcal{V}_\mathrm{d}$作为首选候选域。

\paragraph{传播增益模型}
设载波波长$\lambda$,自由空间段增益
\begin{equation}
 g(\mathbf{a},\mathbf{b})=\Big(\tfrac{\lambda}{4\pi\lVert\mathbf{a}-\mathbf{b}\rVert}\Big)^{2}\,\Xi(\mathbf{a},\mathbf{b}),
\end{equation}
其中$\Xi(\mathbf{a},\mathbf{b})\in(0,1]$可表征阴影/散射等有效折减(在LoS段可取$\Xi\approx 1$)。对单块$N$单元的相位可控RIS,在相位完全对齐的上界近似下,单跳反射的等效增益写作
\begin{equation}
\begin{aligned}
G_{1}(\mathbf{p}_\mathrm{s},\mathbf{r},\mathbf{p}_\mathrm{d}) &= \kappa_{1}\,g(\mathbf{p}_\mathrm{s},\mathbf{r})\,g(\mathbf{r},\mathbf{p}_\mathrm{d}),\\
\kappa_{1} &= \rho\,N^{2},\quad \rho\in(0,1]
\end{aligned}
\end{equation}
多跳$\,m\in\mathbb{N}$时($m\ge 1$),令$\mathbf{u}_{0}=\mathbf{p}_\mathrm{s},\ \mathbf{u}_{m+1}=\mathbf{p}_\mathrm{d},\ \mathbf{u}_{\ell}=\mathbf{r}_{\ell}$为第$\ell$块RIS位置,则
\begin{equation}
\begin{aligned}
G_{m}(\mathbf{u}_{0:m+1}) &= \kappa_{m}\,\prod_{\ell=0}^{m} g(\mathbf{u}_{\ell},\mathbf{u}_{\ell+1}),\\
\kappa_{m} &> 0
\end{aligned}
\end{equation}
汇集了阵元规模、相位合并与硬件效率的上界常数。若源发射功率为$P_{\mathrm{t}}$,目的接收功率为$P_{\mathrm{r}}=P_{\mathrm{t}}\,G_{m}(\cdot)$。

\paragraph{单RIS优化(视域交集 / 走廊约束)}
视域交集模型:
\begin{equation}
\max_{(x,y)\in\mathcal{V}_\mathrm{s}\cap\mathcal{V}_\mathrm{d}}\ \ P_{\mathrm{t}}\,G_{1}\big(\mathbf{p}_\mathrm{s},\mathbf{r}(x,y),\mathbf{p}_\mathrm{d}\big).
\end{equation}
走廊约束模型:
\begin{equation}
\begin{aligned}
\max_{(x,y)\in\mathcal{C}}\ \ &P_{\mathrm{t}}\,G_{1}\big(\mathbf{p}_\mathrm{s},\mathbf{r}(x,y),\mathbf{p}_\mathrm{d}\big)\\
\text{s.t.}\ \ &\mathsf{L}(\mathbf{p}_\mathrm{s},\mathbf{r}(x,y))=\mathsf{L}(\mathbf{r}(x,y),\mathbf{p}_\mathrm{d})=1.
\end{aligned}
\end{equation}

\paragraph{双RIS优化(两端视域 + RIS间LoS)}
令$\mathcal{S}_\mathrm{s}\subseteq\mathcal{V}_\mathrm{s}$、$\mathcal{S}_\mathrm{d}\subseteq\mathcal{V}_\mathrm{d}$为从两端视域中选取的候选子集,则
\begin{equation}
\begin{aligned}
\max_{\mathbf{r}_{1}\in\mathcal{S}_\mathrm{s},\,\mathbf{r}_{2}\in\mathcal{S}_\mathrm{d}}\ \ &P_{\mathrm{t}}\,\kappa_{2}\,g(\mathbf{p}_\mathrm{s},\mathbf{r}_{1})\,g(\mathbf{r}_{1},\mathbf{r}_{2})\,g(\mathbf{r}_{2},\mathbf{p}_\mathrm{d})\\
\text{s.t.}\ \ &\mathsf{L}(\mathbf{r}_{1},\mathbf{r}_{2})=1.
\end{aligned}
\end{equation}

\paragraph{多RIS链优化(3--5跳)}
给定最大跳数$m_{\max}\in\{3,4,5\}$,定义可行序列集合
\begin{equation}
\mathcal{P}_{m}=\Big\{(\mathbf{r}_{1},\ldots,\mathbf{r}_{m}):\ \mathbf{r}_{\ell}\in\mathcal{C},\ \mathsf{L}(\mathbf{u}_{\ell-1},\mathbf{u}_{\ell})=1,\ \ell=1,\ldots,m+1\Big\},
\end{equation}
其中$\mathbf{u}_{0}=\mathbf{p}_\mathrm{s}$、$\mathbf{u}_{\ell}=\mathbf{r}_{\ell}$、$\mathbf{u}_{m+1}=\mathbf{p}_\mathrm{d}$。则多跳优化写作
\begin{equation}
\max_{1\le m\le m_{\max}}\ \ \max_{(\mathbf{r}_{1:m})\in\mathcal{P}_{m}}\ \ P_{\mathrm{t}}\,G_{m}(\mathbf{u}_{0:m+1}).
\end{equation}
如需降低搜索规模,可在$\mathcal{C}$内定义连续评分函数
\begin{equation}
\begin{aligned}
 s(x,y)=\ &\alpha\,H(x,y)-\beta\,\operatorname{dist}\big((x,y),\Pi(\mathbf{p}_\mathrm{s},\mathbf{p}_\mathrm{d})\big)\\
 &{}-\gamma\,\lVert(x,y)-(x_\mathrm{s},y_\mathrm{s})\rVert,
\end{aligned}
\end{equation}
用以选取候选子域(例如取$\mathcal{S}=\{(x,y): s(x,y)\text{居于前}\,K\}$),随后在$\mathcal{S}$上求解上述极大化问题。

\paragraph{与固定通信时程的耦合}
令$\mathcal{T}_{\mathrm{h}}$为“每小时簇内上报”的时间集合,$\mathcal{T}_{\mathrm{d}}$为“每日回传”的时间集合。定义几何路径选择策略$\pi:\mathcal{T}_{\mathrm{d}}\to\bigcup_{m\le m_{\max}}\mathcal{P}_{m}$,使得在每个$t\in\mathcal{T}_{\mathrm{d}}$,
\begin{equation}
\pi(t)\in\arg\max_{(\mathbf{r}_{1:m})\in\bigcup_{m\le m_{\max}}\mathcal{P}_{m}}\ \ P_{\mathrm{t}}\,G_{m}(\mathbf{u}_{0:m+1}),
\end{equation}
并在$t$时隙内按$\pi(t)$对应的几何链路完成CH$\to$目标的回传。注意该策略不影响$\mathcal{T}_{\mathrm{h}}$上的簇内传输,仅作用于每日回传段。

\paragraph{复杂度与可解性(上界)}
在离散候选集$|\mathcal{S}|=N$下,单/走廊模型为$\mathcal{O}(N)$;双RIS在$\mathcal{S}_\mathrm{s}\times$\,$\mathcal{S}_\mathrm{d}$上为$\mathcal{O}(K^{2})$($K=|\mathcal{S}_\mathrm{s}|=|\mathcal{S}_\mathrm{d}|$);多RIS的穷举上界为$\sum_{m\le m_{\max}}\mathcal{O}(N^{m})$,实际可通过候选裁剪与视域约束显著降低到多项式量级的近似求解。

\textbf{协同关系。} 多RIS协作路由构建能量互联网骨干,RIS增强MRC提供簇内高效充能,跨层协调实现端到端优化;本节的数学化几何选址与机会路径机制直接服务于固定回传时程,解决CH$\to$sink/目标的跨地形连通问题,并与簇内每小时上报解耦。

% ===================== RIS 增强 MRC =====================
\subsection{RIS增强MRC:近场透镜与鲁棒耦合}\label{subsec:ris_mrc}
为将RIS用于簇内无线充能,我们采用“RIS近场透镜”思想对线圈-线圈的磁耦合进行增强:
\begin{itemize}
  \item \textbf{近场聚焦:}在工作频段$\omega$下,设计RIS等效表面阻抗$Z_{\mathrm{s}}(\phi)$,使得反射相位满足到接收线圈的等相位球面叠加,提升磁场强度$|\mathbf{H}|$与耦合系数$k$。
  \item \textbf{错位补偿:}把线圈横向/纵向偏移建模为$\Delta x,\Delta z$扰动,基于小扰动灵敏度$\partial \eta/\partial \Delta$在线/离线调整$Z_{\mathrm{s}}$表面编程,以维持效率$\eta$。
  \item \textbf{能量-信息共存:}充能时隙采用占空比调度,保持簇内汇聚与测量上报信道的正交或低互扰(如TDMA)。
\end{itemize}

% ===================== 网络层设计 =====================
\subsection{网络层设计:定时上报与能量路由}\label{subsec:net_design}
结合地形感知的RIS选址结果,网络层按固定的轻量时程运行:
\begin{itemize}
  \item \textbf{簇内上报(每小时):}传感器$\to$CH采用TDMA/CSMA小包上报;CH做数据融合并缓存;该阶段不依赖RIS几何路径。
  \item \textbf{每日回传(一次/日):}在$\mathcal{T}_{\mathrm{d}}$时隙,按图~\ref{fig:ris_dist} 所示的几何路径从CH沿RIS链路回传至sink/目标;路径由第~\ref{subsec:ris_placement_mech} 的最优化得到。
  \item \textbf{无线传能逻辑:}白天RF1/2(山阳面)优先作为供能端,沿几何路径或就近RIS对RF3--RF6(山阴面)补能;能量分配遵循“最大覆盖/最低保障”策略(可选),并受供能预算约束。
  \item \textbf{自适应修正:}周期性测量LoS与阴影变化(如雨雾/植被),当链路衰减超阈值时触发重算候选或切换到备用路径。
\end{itemize}

\paragraph{与第3章的衔接}
- 全局层:第~\ref{sec:carrauntoohil} 节给出 DEM/视域建模与参数来源;相关几何机制以该模型为输入;
- 簇内层:第~\ref{sec:cluster_scene} 节给出 RIS 对 MRC 的增益建模,可在局部层用于远距与错位补偿(与全局层互补)。
