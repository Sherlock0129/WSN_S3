\section{讨论}\label{sec:discussion}
本节从系统性解读、工程设计启示、复杂度与可部署性、局限与威胁、外推性与未来工作六个维度,对第~\ref{sec:experiments} 节的结果与第~\ref{sec:conclusion} 节的结论进行归纳与反思,力求将“全局几何—局部近场—跨层协调”的三层机制联结为可落地的工程范式。

\paragraph{关键发现的综合解读}
- 全局(G,远场多RIS):在越岭 NLoS 场景,通过受走廊与视域约束的多跳反射路径,CH 侧接收功率与能量可达率显著抬升。双/三跳在复杂地形中呈现“性价比最优”的拐点,过多跳数边际收益递减且控制开销上升。RIS 安装高与走廊半宽存在较稳定的最佳区间(第~\ref{sec:experiments} 节)。
- 局部(L,近场透镜 + MRC):RIS 作为能量透镜在中/大错位与倾斜扰动下显著提升 MRC 效率,等效“有效补能体积”外扩,降低对机械精度与定位误差的敏感度,实现对部署瑕疵的容错(见第~\ref{sec:experiments} 节局部层)。
- 跨层(C,联合调度):在 G+L 的物理增益之上,引入帧级时隙/功率/RIS 模式的联合调度(如 ALDP/EETOR 族)后,能量短缺率进一步下降,SOC 分布更均衡,FND/HND 延长。相位量化在 $b\ge2$ 时已接近饱和,RIS 更新周期可日/小时级少量更新即可维持收益,显示对有限控制预算的鲁棒性。

\paragraph{设计启示与工程建议}
- 资源优先级与预算分配:在固定控制/成本预算下,优先增加 RIS 数量与合理的安装高度,随后再考虑提高相位量化比特;当候选密度受限时,适度放宽走廊半宽以提升几何可连通性。
- 控制面与状态摘要:以“低开销能量状态摘要”驱动跨层决策,将需求感知与多 RIS 能量路由绑定;RIS 配置与调度指令采用分层/增量编码,日级为主、小时级为辅,兼顾收益与开销。
- 近场布局与容错:簇内优先部署少量“主导 RIS 面板”对准 CH—节点典型相对几何,配合 MRC 查表修正以覆盖错位分布的中位与尾部;对超低功耗节点可用离线生成的轻量启发式策略替代在线求解。
- 供能与运维:固定 RIS 建议配套微型光伏或能量驯化模块,移动 RIS(UAV/地面机器人)用于临时补盲与季节性再标定;预留快速现场标定与健康度监测接口以降低长期漂移带来的维护成本。

\paragraph{复杂度、可部署性与实现路径}
- 算法复杂度:跨层优化在帧级引入求解负担,可采用“离线候选裁剪 + 在线轻量决策”的两阶段策略;束搜索宽度与候选规模可按地形复杂度自适应裁剪,以控制在线时延(第~\ref{sec:experiments} 节复杂度指标)。
- 数据与建模依赖:几何连通性强依赖 DEM 质量与遮阴估计精度;建议在部署前进行稀疏测绘与视域验证,并对关键反射段进行小样本现场标定,以提高模型—现实一致性。
- 软硬件协同:RIS 相位控制、MRC 驱动与 WSN MAC 需共享统一的时间参考与极简控制面;建议采用边缘控制器汇聚 RIS 指令并进行状态压缩,减少节点端的协议复杂度。

\paragraph{局限性与威胁}
- 控制与标定开销:RIS 数量增加带来相位标定、插入损耗评估与温漂补偿的额外开销;当环境快速变化(地形积雪、植被季节性变化、临时遮挡)时,需要更快的重配置节律以维持反射链路质量。
- 模型与硬件不完备性:近场耦合与多反射通道存在非理想因素(互耦、量化误差、面板间时延差);DEM 精度与太阳几何估计误差会传导至可达性评估,可能低估极端天气下的波动。
- 安全与鲁棒性:RIS 控制面存在被篡改/干扰的潜在风险,应配套轻量认证与回退策略;跨层调度的误报/漏报会影响能量均衡,需要异常检测与保护带(guard band)。
- 合规与干扰:RF 输能需满足当地辐射与 EIRP 法规,注意与既有通信系统的频谱隔离与旁瓣控制;移动 RIS 应遵循空域/作业安全规范。

\paragraph{适用范围与外推性}
- 场景外推:所提“几何机会 + 近场容错 + 跨层协调”的范式可推广至城市峡谷、风场/林区传感、农田与水域监测等存在 NLoS 约束与错位不确定性的环境;当反射走廊稀缺或 DEM 粗糙时,应优先采用 L+C 以获得稳健收益。
- 技术融合:框架与 SWIPT/波形共设兼容,可将 RIS 相位控制与能量—信息联合度量协同优化;在工业与医疗等边缘场景,能量信息协同的思路同样适用,但需更严格的安全与可靠性保障。

\paragraph{未来工作}
- 学习增强的相位/路由:引入数据驱动的相位码本与多跳路由选择,结合不确定性估计以提升在动态环境下的适应性与样本效率。
- 闭环标定与自监测:构建“测量—校正—验证”闭环,在线估计插入损耗与相位漂移,实现 RIS/MRC 的长期自校准。
- 联合波形与协议共设:探索与 SWIPT/反射调制的协同,面向能量—数据双重目标的帧级协议共设。
- 安全与治理:设计轻量认证、指令完整性保护与异常检测机制,形成可审计的控制面;评估跨域部署的合规路径。
- 机动平台试验:开展 UAV/地面机器人挂载 RIS 的野外原型验证,量化移动性带来的部署灵活性与链路稳健性收益。

综上,本文验证了“以低开销能量状态摘要驱动的跨层优化 + 多 RIS 远场能量路由 + 近场透镜增强 MRC”的统一范式在复杂地形 WSN 的工程可行性与稳健收益;同时也明确了其在控制、标定与安全上的现实边界,为后续走向规模化与长期运行提供了可执行的路线图。