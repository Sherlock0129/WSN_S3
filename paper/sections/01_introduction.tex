\section{引言}

大规模无线传感器网络(WSN)正广泛用于环境监测、精准农业、森林防火、边境/基础设施巡检等任务。然而,真实户外地形——如山地、森林与峡谷——带来严峻的供能挑战:
(i)太阳能在阴影遮挡与高纬冬季低太阳高度角条件下稳定性差;
(ii)长距离射频无线输能(RF WPT)在非视距(NLOS)与越岭传播中路径损耗与多径衰落显著;
(iii)磁共振耦合(MRC)WPT对线圈错位、倾斜与相对运动高度敏感,限制了近场补能的鲁棒性。海洋性气候背景下的爱尔兰西南部山地尤为典型:频繁的云量变化、湿润大气与陡峭地形叠加,使分布式WSN的长期、可靠供能更具挑战。

可重构智能表面(RIS)作为可编程超表面,能够在低硬件与能耗开销下对电磁波前进行定制重构,已被证明可用于增强覆盖、聚焦能量与抑制干扰。尽管如此,现有研究多聚焦于室内或简化户外场景、单面板配置与通信导向优化,针对“复杂地形NLOS条件下的多RIS协作能量路由”“RIS增强的MRC近场聚焦鲁棒性”“跨层(物理—拓扑—应用)能量流一体化优化”仍缺乏系统化方法与经实地地形约束的验证。

\textbf{核心科学问题:}如何设计、建模与优化一个\emph{可扩展的分层RF--RIS--MRC无线输能系统},使其能在复杂地形与NLOS环境中可靠路由能量、兼顾远距离RF输能与近场MRC补能,并在统一框架下实现端到端能量流的协同与鲁棒运行?

为回答上述问题,本文提出一种\emph{分层、可扩展}的混合无线输能(WPT)架构,以RIS为关键可编程中介,构建“RIS + WPT + WSN”的一体化生态, 并在爱尔兰西南部一处山峰($51^\circ 59' 40''\,\mathrm{N}$,$9^\circ 42' 02''\,\mathrm{W}$)的真实地形上进行场景化仿真与验证。架构包含三个层次:
- \textbf{全局层}:多RIS面板通过可编程反射形成可控的多跳能量通道,跨越山脊/谷地实现远距离RF能量路由;结合高分辨率数字高程模型(DEM)进行视距/遮挡判定、绕射/反射路径搜索与RIS选址优化。
- \textbf{局部(簇)层}:在簇内利用RIS作为“近场能量透镜”,与MRC协同提升磁耦合效率、扩大有效补能体积,并显著增强对线圈错位、倾斜与运动的容忍度。
- \textbf{网络层}:在统一的RF--RIS--MRC--WSN能量流模型下,进行端到端能量预算、跨层协同调度与鲁棒性约束(如天气型、负载波动与节点移动)的联合优化。

\begin{figure}[!t]
  \centering
  \subfloat[复杂山地中的 RIS 协作能量路由示意。向阳坡(Adret)部署 RF 源,经多个 RIS 反射跨越山脊,将能量送达背阴侧(Nightside)的簇头;红色虚线表示入射波,蓝色虚线表示反射波。]{%
    \includegraphics[width=.98\linewidth]{\detokenize{sections/figures/S3场景示意图1.png}}\label{fig:intro_scene_a}}
  \\
  \subfloat[研究区与太阳几何。左:爱尔兰地形与研究区位置;右上:北半球日轨迹,冬至太阳高度角最低,背阴面长时间处于阴影;右下:研究区山峰实景。]{%
    \includegraphics[width=.98\linewidth]{\detokenize{sections/figures/S3场景示意图2.png}}\label{fig:intro_scene_b}}
  \vspace{-0.5em}
  \caption{引言场景示意。(a)RIS 多跳反射在 NLoS 条件下构建可控能量通道;(b)研究区与冬至太阳几何强调背阴面供能困难,动机 RIS 辅助路由与近场透镜以弥补太阳能不稳定与直射受阻。}
  \label{fig:intro_scene}
\end{figure}

如图\ref{fig:intro_scene} 所示:图\subref{fig:intro_scene_a} 给出了在起伏地形中由多块 RIS 构成的“人工可控路径”,将向阳坡的 RF 能量跨越山脊路由至背阴侧簇头;图\subref{fig:intro_scene_b} 标示了研究区位置与太阳在不同季节的日轨迹,冬至时低太阳高度角导致背阴面长时间阴影,传统太阳能可靠性下降。这一现实背景直接动机了本文的分层设计:全局层通过多 RIS 多跳抬升可达性,局部层以 RIS 透镜增强 MRC 的抗错位充能,网络层进行跨层能量流协同以应对天气型与负载波动。

与传统文献相比,本文在“思想与方法”上兼顾原有分层体系结构与真实地形要素:在全局层引入基于地形的多跳能量路由与RIS布局;在局部层刻画RIS增强MRC的近场聚焦鲁棒性;在网络层实现跨层能量流一体化优化与可部署性评估,并在目标山地场景中进行基于实地约束的高分辨率仿真。

本文的主要贡献包括:
\begin{itemize}[leftmargin=*]
  \item \textbf{多RIS协作的NLOS能量路由:}提出面向复杂山地的多跳RIS能量通道构建方法,结合DEM进行遮挡分析与路径搜索,实现越岭/跨谷的远距离RF WPT;
  \item \textbf{RIS增强的MRC近场能量透镜:}建立近场聚焦与错位鲁棒性模型,给出面向移动或定位误差的补能体积(charging volume)设计与评估;
  \item \textbf{统一的跨层能量流协调:}构建RF--RIS--MRC--WSN端到端能量流模型与优化框架,联合考虑天气型、业务负载与节点拓扑,提升系统可用性与能效;
  \item \textbf{实地地形约束的场景化仿真:}在爱尔兰西南部山峰($51^\circ 59' 40''\,\mathrm{N}$,$9^\circ 42' 02''\,\mathrm{W}$)进行高分辨率仿真,量化多RIS路由增益、近场透镜鲁棒性与端到端能量供给能力,并与\emph{无RIS}、\emph{单RIS}与\emph{仅MRC}等基线进行对比。
\end{itemize}

本文结构如下:第\ref{sec:related}节综述RIS辅助RF WPT、超表面增强MRC与WSN能量管理及地形传播相关研究;第\ref{sec:model}节给出系统模型、研究区与数据(含DEM与RIS/节点布局先验);第\ref{sec:mechanism}节阐述多RIS能量路由、RIS增强MRC与跨层能量流优化的方法;第\ref{sec:experiments}节在真实地形场景中进行仿真与对比评估;第\ref{sec:discussion}节讨论可扩展性、鲁棒性与部署要点;第\ref{sec:conclusion}节总结全文并展望未来工作。