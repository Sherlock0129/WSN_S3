\section{引言}

大规模无线传感器网络(WSN)在环境监测、农业、森林防火、边境监控和工业设备管理等应用中广泛部署。然而,在真实地形环境——如山地、森林、峡谷——中,WSN节点面临严峻的能量供应挑战。传统能量获取方式(如电池和太阳能采集)在阴影遮挡、非视距(NLOS)场景下不可靠;长距离射频无线输能(RF WPT)在障碍地形中路径损耗严重;磁共振耦合(MRC)WPT对线圈错位、倾斜和移动极为敏感,限制了其在户外应用中的鲁棒性。

可重构智能表面(RIS)技术的进展为解决上述问题提供了新机遇。RIS作为可编程超表面,能够主动调控电磁波传播,既可用于远距离能量传输,也可用于近场能量聚焦。然而,现有研究主要关注单一RIS面板在室内场景中的应用,对多RIS协作能量路由、复杂地形环境下的能量传输以及RIS增强的MRC近场聚焦等关键问题缺乏系统性研究。

\textbf{核心科学问题:}如何设计、建模和优化一个可扩展的分层RF--RIS--MRC无线输能系统,使其能够在复杂的非视距环境中可靠地传输能量,并高效地为大规模WSN部署供能,同时在统一框架内协调长距离RF能量路由和近场磁共振充能?

本文提出一种可扩展、可扩展的分层混合无线输能(WPT)架构,由可重构智能表面(RIS)支撑,构建RIS + WPT + WSN一体化生态系统。该架构包含三个层次:\textbf{全局层},多个RIS面板形成可控的多跳反射能量路径,实现跨山地或障碍物的长距离RF能量路由;\textbf{局部(簇)层},可编程超表面作为近场能量透镜,增强磁耦合效率和对错位的容忍度;\textbf{网络层},统一的RF--RIS--MRC--WSN能量流模型协调端到端能量传输。

本文的主要贡献包括:
\begin{itemize}[leftmargin=*]
  \item 提出多RIS协作的NLOS环境能量路由方法,解决复杂地形下的长距离能量传输问题。
  \item 设计RIS增强的MRC近场能量透镜机制,提升磁耦合效率并增强对线圈错位的鲁棒性。
  \item 构建统一的跨层能量流协调框架,实现RF--RIS--MRC--WSN系统的端到端能量优化。
  \item 建立可扩展、可扩展的WPT架构,支持大规模WSN部署,并兼容无人机载RIS、移动机器人和6G物联网等未来智能能源基础设施。
\end{itemize}

本文结构如下:第\ref{sec:related}节综述RIS辅助RF WPT、超表面增强MRC以及WSN能量管理相关研究;第\ref{sec:model}节阐述系统建模与关键科学问题;第\ref{sec:mechanism}节详述多RIS RF能量路由、RIS增强MRC能量透镜和跨层能量流协调的技术路线;第\ref{sec:experiments}节给出实验设计与验证指标;第\ref{sec:discussion}节讨论系统可扩展性与扩展性;第\ref{sec:conclusion}节总结全文并展望未来工作。
