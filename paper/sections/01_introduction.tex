\section{引言}

山地环境中的能量与气候过程在空间与时间上高度异质,受地形起伏、坡度坡向、天空视域以及大气条件的共同调制。这种复杂性在中高纬度冬季尤为突出:太阳高度角降低、直射路径延长、地形遮蔽效应加强,导致短波辐射的时空分布剧烈分化,并通过地表能量平衡影响近地层温度、土壤冻结—解冻、植被光能利用与水文响应。对于海洋性气候背景下的爱尔兰西南部山地而言,强盛的西风带、频繁的云变性与陡峭地形叠加,使得在极端太阳几何条件(如冬至日)下定量刻画辐射与微气象场的精细结构既具科学挑战,也具备明确的应用价值(如生态过程评估、地表过程模拟、可再生能源选址与高程带风险管理)。

尽管已有研究从遥感统计、经验参数化与中尺度模式等层面揭示了地形对辐射和微气候的调制效应,然而在米—十米级地形控制下的瞬时与日内时段尺度过程仍存在显著认识空缺,主要体现在:1)低太阳高度角条件下,复杂地形阴影、坡面法向入射角与天空散射的相对贡献及其随时间的快速跃迁尚缺乏系统、可验证的刻画;2)海洋性大气条件与地形相互作用下的“晴空窗口”与薄云扩散对局地辐射场的可预报性不足;3)模型在实地高分辨率观测约束下的可迁移性与不确定性来源尚未充分量化。因此,建立一种以实测为约束、面向复杂地形与极端太阳几何条件的高分辨率物理仿真框架,是推进山地环境机理认识与服务应用转化的关键路径。

\textbf{核心科学问题:}如何在海洋性气候与复杂地形叠加、低太阳高度角的条件下,于米级空间分辨率稳定、可解释地重建短波辐射的直达、散射与地形反照三分量及其地形调制效应,并在实地观测约束下实现偏差归因与跨地点迁移?

为回答上述问题,本文以爱尔兰西南部一处山峰($51^\circ 59' 40''\mathrm{N}$,$9^\circ 42' 02''\mathrm{W}$)为试验场,围绕冬至期的极端太阳几何条件,提出一个\emph{分层的仿真—观测一体化框架},借鉴分层设计思想以增强可扩展性与复用性:\textbf{全局层},基于太阳几何与大气透过率(含云量/气溶胶清洁度)给出顶端入射场与辐射分解先验;\textbf{地形层},结合数字高程模型(DEM)与地形几何量(坡度、坡向、天空视域因子)进行地形遮蔽判定与光路追踪,解析阴影锋面的时空推进;\textbf{观测—校准层},利用实地布设的辐射与气象观测对关键参数进行同化与校准,开展敏感性分析与不确定性量化,并评估不同坡面和高程带的模型可迁移性。

本文的主要贡献包括:
\begin{itemize}[leftmargin=*]
  \item 提出面向冬至极端太阳几何条件的高分辨率地形—辐射耦合框架,系统描述复杂地形下短波辐射三分量的主导控制机制与时空模式;
  \item 发展阴影锋面时空追踪与可视化方法,刻画低太阳高度角下阴影跃迁对辐射场的瞬态影响;
  \item 在实地观测约束下开展参数敏感性与不确定性分解,明确大气透过率、天空散射与地形遮蔽判定对模拟偏差的贡献;
  \item 形成可复用的建模与校准流程,为海洋性中纬度山地的生态水文模拟、微气候评估与工程选址提供方法学支撑,并为季节与天气型推广奠定基础。
\end{itemize}

本文结构如下:第\ref{sec:related}节综述复杂地形辐射建模、地形参数化与观测校准相关研究;第\ref{sec:model}节介绍研究区、数据与整体框架;第\ref{sec:mechanism}节给出地形遮蔽判定、辐射分解与不确定性分析的方法学细节;第\ref{sec:experiments}节展示冬至情景的高分辨率仿真与实测验证;第\ref{sec:discussion}节讨论敏感性、可迁移性与应用前景;第\ref{sec:conclusion}节总结全文并展望未来工作。