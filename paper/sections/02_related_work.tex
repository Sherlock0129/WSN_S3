\section{相关工作}\label{sec:related}

\textbf{RIS 辅助 RF 无线输能(WPT)。} 近年来,RIS 被用于重构传播环境以提升能量覆盖与聚焦能力。多数工作集中在单块 RIS、近似 LoS 或室内场景;在复杂地形与 NLoS 条件下,关于多 RIS 级联反射、多跳能量通道构建、越岭/跨谷路由以及与数字高程模型(DEM)结合的研究仍较有限。典型议题包括:反射/绕射/散射的综合通道建模,有限相位量化与单元耦合导致的非理想增益,RIS 规模与阵列因子对能量指向性的影响,以及 RIS 的选址与姿态优化。

\textbf{SWIPT 与波形/频段设计。} 面向同时信息与能量传输(SWIPT),多音/多载波、峰均比(PAPR)控制与非线性整流模型驱动的波形共设计可显著提升整流效率;当与 RIS 结合时,需要联合选择频段与波形,并协调反射相位时频调度,实现能量/信息的空间分离与干扰管理。对低频段(如亚 GHz)越岭传播与高频段(毫米波)指向性之间的权衡仍是开放问题。

\textbf{极化分集与阵列工程。} 户外部署中存在姿态不确定与极化失配。通过双极化单元、交错阵列与可切换极化 RIS 结构,可降低极化失配损失并增强对姿态变化的鲁棒性;单元间耦合抑制与口径/填充因子设计对能量指向性与副瓣控制同样关键。

\textbf{多 RIS 协作与 NLoS 反射链路。} 多 RIS 通过级联反射形成“人工可控路径”,可在遮挡场景下延伸 WPT 可达性。关键挑战包括:在地形约束下进行视距/遮挡判定与候选反射点筛选;在路径损耗、反射增益与控制开销之间权衡;以及在环境变化(植被/湿度/临时遮挡)下保持鲁棒性与可重配置。移动或无人机(UAV)载 RIS 提供按需部署与临时补盲能力,但对能量成本、悬停稳定性与安全提出新要求。

\textbf{RIS 硬件与控制开销。} 实际 RIS 存在相位量化、幅度损耗、单元间耦合与频率色散等非理想因素;大规模面板需要层级控制、低开销反馈与自校准。自供电 RIS(能量采集 + 超低功耗控制)、反射/透射一体化超表面与片上/片外控制器协同,是面向野外长期运行的可行技术路径。

\textbf{RIS 增强的近场能量聚焦(与 MRC 协同)。} 近场能量传输对几何错位高度敏感。可编程超表面可作为“能量透镜”重塑近场场型,提升耦合系数并扩展有效充电体积,对线圈的平移/倾斜更鲁棒。开放问题包括:聚焦强度、效率与错位鲁棒性的三者权衡;多线圈/多接收端的同时供能调度;以及面向位置不确定性的自适应相位控制与快速重构。

\textbf{WSN 能量管理与跨层协同。} 传统 WSN 能量管理多侧重能量采集与占空优化,难以利用可控的能量路由资源。将 RIS 视作网络级“能量路由器”,并与 RF 远场输能、MRC 近场补能协同,形成跨层能量流优化框架,可面向任务负载、地形约束与部署成本进行端到端权衡;与 SWIPT 联动可进一步在频谱/空间维度做联合分配。

\textbf{面向真实地形的部署与传播。} 在山地等复杂地形中,DEM 支持的视距/遮挡与候选反射点搜索、地貌驱动的 RIS 选址与姿态优化、以及与气象(风/云量/湿度)协同的时变传播,将决定系统可用性与能效。与通信导向 RIS 文献相比,能量输运需要显式关注端到端功率预算、硬件约束与维护/控制开销。

\textbf{研究缺口。} 归纳来看,仍缺少:(i)\emph{多 RIS 协作下的 NLoS 能量互联网框架},可在 DEM 约束下构建/优化多跳反射通道;(ii)\emph{RIS 增强的近场透镜与 MRC 的系统性结合},面向错位鲁棒与多接收端充电;(iii)\emph{统一的跨层能量流优化},在远/近场、频段/波形、控制与硬件约束间联合权衡;(iv)\emph{面向真实地形的场景化评估},兼顾选址、姿态、移动 RIS 与长期运行的控制/能量开销。本文针对上述缺口,提出分层 RF--RIS--MRC 架构,并在爱尔兰山地实地地形上进行高分辨率仿真验证。