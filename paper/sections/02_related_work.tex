\section{相关工作}\label{sec:related}

\textbf{RIS辅助RF输能。} 现有研究多聚焦单块RIS的室内场景,针对多RIS协同、复杂地形NLOS反射链路的能量路由较少,难以支撑远距离跨障碍供能。现有工作未将RIS视为能量路由网络中的主动节点,缺乏构建能量互联网(Energy Routing Network)的系统性框架。

\textbf{超表面增强MRC。} 可编程超表面可重塑近场磁场分布,但多数工作基于固定结构,抗偏移能力有限,且与实际WSN部署结合不足。现有研究未充分考虑复杂地形环境下线圈错位、倾斜等实际部署问题。

\textbf{WSN能量管理。} 传统能量采集、SWIPT、值班调度假设能量流不可控,未考虑RF--MRC联动,RIS很少作为主动路由资源。现有方法缺乏系统级整合,难以实现RIS + WPT + WSN一体化生态系统。

\textbf{信息时效。} AOI/AOEI用于状态更新,但将能量状态新鲜度与能量路由/充能联动的研究尚缺。

\textbf{研究缺口:} 现有研究存在以下关键缺口:(i) \textbf{缺乏RIS + WPT + WSN一体化生态系统:}缺少系统级整合框架,未能实现长期供能、自组织能力和工程实用路线的统一;(ii) \textbf{缺乏跨山体能量反射方案:}未针对山地环境的特殊需求(如阴坡补盲、NLOS能量覆盖)设计专门的能量传输方案;(iii) \textbf{缺乏多RIS协同能量路由系统:}未将RIS视为能量节点构建能量互联网框架,缺乏多跳反射链路和能量路径选择的系统性方法;(iv) \textbf{缺乏统一框架:}缺少同时覆盖多跳RIS基于NLOS的RF能量路由、RIS增强MRC的鲁棒聚焦、信息新鲜度与能量共享协同优化的统一框架。

本文针对上述缺口,提出可扩展的分层RF--RIS--MRC无线输能系统,在统一框架内协调长距离RF能量路由和近场磁共振充能,构建RIS + WPT + WSN一体化生态系统。
