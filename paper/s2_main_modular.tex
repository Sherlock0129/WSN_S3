\documentclass[journal]{IEEEtran}

% *** MISC UTILITY PACKAGES ***
\ifCLASSINFOpdf
  \usepackage[pdftex]{graphicx}
  \graphicspath{{sections/figures/}}
  \DeclareGraphicsExtensions{.pdf,.jpeg,.png}
\else
  \usepackage[dvips]{graphicx}
  \graphicspath{{sections/figures/}}
  \DeclareGraphicsExtensions{.eps}
\fi

% *** MATH PACKAGES ***
\usepackage{amsmath,amssymb,amsfonts}
\usepackage{mathtools} % extends amsmath
\usepackage{algorithmic}

% Allow equation breaks across lines/columns per IEEEtran guidance
\interdisplaylinepenalty=2500
\allowdisplaybreaks[3]

% *** SPECIALIZED LIST PACKAGES ***
\usepackage{enumitem}

% *** ALIGNMENT PACKAGES ***
\usepackage{array}
\usepackage{booktabs}  % 三线表支持
\usepackage{multirow}  % 跨行表格支持
\usepackage{makecell}  % 表格单元格换行

% *** SUBFIGURE PACKAGES ***
\usepackage[caption=false,font=footnotesize]{subfig}

% *** FLOAT PACKAGES ***
\usepackage{stfloats}

% *** PDF, URL AND HYPERLINK PACKAGES ***
\usepackage{url}
\usepackage{hyperref}

% *** TYPOGRAPHY ***
\usepackage[final]{microtype} % better kerning/expansion; reduces overfull boxes

% *** CHINESE SUPPORT ***
\usepackage[UTF8]{ctex}

% *** ADDITIONAL PACKAGES ***
\usepackage{siunitx}
\usepackage{textcomp}
\usepackage{xcolor}
\usepackage{tikz}
\usetikzlibrary{positioning, shapes.geometric, arrows}
\usepackage{mdframed}  % 可分页的背景框
\hyphenation{op-tical net-works semi-con-duc-tor}

\begin{document}

\title{基于可重构智能表面的可扩展分层无线输能系统:面向非视距(NLOS)场景下的WSN的能量传输解决方案}

\author{作者姓名~\IEEEmembership{Member,~IEEE,}
        和~合作者姓名~\IEEEmembership{Member,~IEEE}%
\thanks{通信作者单位与邮箱。}%
\thanks{其他作者单位与邮箱。}%
\thanks{稿件收到与修订日期。}}

% The paper headers
\markboth{IEEE Transactions on XXX,~Vol.~XX, No.~X, Month~YYYY}%
{Shell \MakeLowercase{\textit{et al.}}: Information--Energy Dual System for Energy Sharing in WSN}

\maketitle

\begin{abstract}
大规模无线传感网(WSN)在山地、森林、工业等复杂环境中面临可持续供能难题:遮挡导致太阳能不稳定,NLOS 场景使射频无线输能(RF WPT)路径损耗严重,磁共振耦合(MRC)对线圈偏移高度敏感。本文提出基于可重构智能表面(RIS)的可扩展分层混合输能体系:全局层利用多块 RIS 形成可控多跳反射功率通路,实现跨障碍远距离 RF 供能;簇内层以可编程超表面作为近场能量透镜,提升 MRC 耦合效率和抗偏移能力;网络层构建统一的 RF--RIS--MRC--WSN 能量流模型以协同端到端输能。体系兼具可扩展性(支持大规模 WSN 拓展)与可扩充性(可接入无人机/机器人挂载 RIS 与未来 6G IoE 能源基础设施),结合理论建模、电磁仿真、原型实现与系统验证,面向工程落地。
\end{abstract}

\begin{IEEEkeywords}
无线传感网,RIS,多跳射频无线输能,磁共振耦合
\end{IEEEkeywords}

\IEEEpeerreviewmaketitle

% ========== 各章节通过 \input 引用 ==========
\section{引言}

大规模无线传感器网络(WSN)在环境监测、农业、森林防火、边境监控和工业设备管理等应用中广泛部署。然而,在真实地形环境——如山地、森林、峡谷——中,WSN节点面临严峻的能量供应挑战。传统能量获取方式(如电池和太阳能采集)在阴影遮挡、非视距(NLOS)场景下不可靠;长距离射频无线输能(RF WPT)在障碍地形中路径损耗严重;磁共振耦合(MRC)WPT对线圈错位、倾斜和移动极为敏感,限制了其在户外应用中的鲁棒性。

可重构智能表面(RIS)技术的进展为解决上述问题提供了新机遇。RIS作为可编程超表面,能够主动调控电磁波传播,既可用于远距离能量传输,也可用于近场能量聚焦。然而,现有研究主要关注单一RIS面板在室内场景中的应用,对多RIS协作能量路由、复杂地形环境下的能量传输以及RIS增强的MRC近场聚焦等关键问题缺乏系统性研究。

\textbf{核心科学问题:}如何设计、建模和优化一个可扩展的分层RF--RIS--MRC无线输能系统,使其能够在复杂的非视距环境中可靠地传输能量,并高效地为大规模WSN部署供能,同时在统一框架内协调长距离RF能量路由和近场磁共振充能?

本文提出一种可扩展、可扩展的分层混合无线输能(WPT)架构,由可重构智能表面(RIS)支撑,构建RIS + WPT + WSN一体化生态系统。该架构包含三个层次:\textbf{全局层},多个RIS面板形成可控的多跳反射能量路径,实现跨山地或障碍物的长距离RF能量路由;\textbf{局部(簇)层},可编程超表面作为近场能量透镜,增强磁耦合效率和对错位的容忍度;\textbf{网络层},统一的RF--RIS--MRC--WSN能量流模型协调端到端能量传输。

本文的主要贡献包括:
\begin{itemize}[leftmargin=*]
  \item 提出多RIS协作的NLOS环境能量路由方法,解决复杂地形下的长距离能量传输问题。
  \item 设计RIS增强的MRC近场能量透镜机制,提升磁耦合效率并增强对线圈错位的鲁棒性。
  \item 构建统一的跨层能量流协调框架,实现RF--RIS--MRC--WSN系统的端到端能量优化。
  \item 建立可扩展、可扩展的WPT架构,支持大规模WSN部署,并兼容无人机载RIS、移动机器人和6G物联网等未来智能能源基础设施。
\end{itemize}

本文结构如下:第\ref{sec:related}节综述RIS辅助RF WPT、超表面增强MRC以及WSN能量管理相关研究;第\ref{sec:model}节阐述系统建模与关键科学问题;第\ref{sec:mechanism}节详述多RIS RF能量路由、RIS增强MRC能量透镜和跨层能量流协调的技术路线;第\ref{sec:experiments}节给出实验设计与验证指标;第\ref{sec:discussion}节讨论系统可扩展性与扩展性;第\ref{sec:conclusion}节总结全文并展望未来工作。

\section{相关工作}\label{sec:related}

\textbf{RIS辅助RF输能。} 现有研究多聚焦单块RIS的室内场景,针对多RIS协同、复杂地形NLOS反射链路的能量路由较少,难以支撑远距离跨障碍供能。现有工作未将RIS视为能量路由网络中的主动节点,缺乏构建能量互联网(Energy Routing Network)的系统性框架。

\textbf{超表面增强MRC。} 可编程超表面可重塑近场磁场分布,但多数工作基于固定结构,抗偏移能力有限,且与实际WSN部署结合不足。现有研究未充分考虑复杂地形环境下线圈错位、倾斜等实际部署问题。

\textbf{WSN能量管理。} 传统能量采集、SWIPT、值班调度假设能量流不可控,未考虑RF--MRC联动,RIS很少作为主动路由资源。现有方法缺乏系统级整合,难以实现RIS + WPT + WSN一体化生态系统。

\textbf{信息时效。} AOI/AOEI用于状态更新,但将能量状态新鲜度与能量路由/充能联动的研究尚缺。

\textbf{研究缺口:} 现有研究存在以下关键缺口:(i) \textbf{缺乏RIS + WPT + WSN一体化生态系统:}缺少系统级整合框架,未能实现长期供能、自组织能力和工程实用路线的统一;(ii) \textbf{缺乏跨山体能量反射方案:}未针对山地环境的特殊需求(如阴坡补盲、NLOS能量覆盖)设计专门的能量传输方案;(iii) \textbf{缺乏多RIS协同能量路由系统:}未将RIS视为能量节点构建能量互联网框架,缺乏多跳反射链路和能量路径选择的系统性方法;(iv) \textbf{缺乏统一框架:}缺少同时覆盖多跳RIS基于NLOS的RF能量路由、RIS增强MRC的鲁棒聚焦、信息新鲜度与能量共享协同优化的统一框架。

本文针对上述缺口,提出可扩展的分层RF--RIS--MRC无线输能系统,在统一框架内协调长距离RF能量路由和近场磁共振充能,构建RIS + WPT + WSN一体化生态系统。

\section{系统建模}\label{sec:model}

本章从“节点建模”和“场景建模”两个层面给出我们所建立的全部模型,覆盖能量、近场耦合/超表面、计算通信负载、存储与控制开销等,为后续机制设计与优化奠定基础。

\subsection{节点建模}\label{subsec:node_model}
本节针对四类节点——传感器节点(Sensor)、可重构智能反射面(RIS)、RF 簇头(Cluster Head, CH)与 Sink(信息汇聚与调度)——分别给出能量、耦合/物理、计算通信与存储控制等模型。

\subsubsection{传感器节点(Sensor)}
\paragraph{(a) 能量模型:光伏获能 + 用能 + 向 CH 的 MRC 送能}
传感器节点通过太阳能获能,并以近场 MRC 方式向 CH 反向送能。设每帧周期为 \(T_f\)。
- 光伏获能:
\begin{equation}
\begin{aligned}
P_{\text{PV}}(t) &= \eta_{\text{pv}}\,A_{\text{pv}}\,G_t(t)\,\Xi(\psi,\theta_{\odot},t),\\
E_{\text{PV}} &= \int_{t}^{t+T_f} P_{\text{PV}}(\tau)\,\mathrm{d}\tau
\end{aligned}
\end{equation}
其中 \(\eta_{\text{pv}}\) 为组件效率,\(A_{\text{pv}}\) 为面积,\(G_t\) 为到达水平面的总辐照度,\(\Xi\in[0,1]\) 为入射角/遮挡修正项(含地形/建筑遮阴)。
- 本地用能:
\begin{equation}
\begin{aligned}
E_{\text{use}} &= P_{\text{sense}}t_{\text{sense}} + P_{\text{comp}}t_{\text{comp}} + P_{\text{tx}}^{\text{RF}}t_{\text{tx}}\\
&\quad + P_{\text{rx}}t_{\text{rx}} + P_{\text{idle}}t_{\text{idle}}
\end{aligned}
\end{equation}
- 向 CH 的 MRC 送能:节点以线圈驱动功率 \(P_{u\to k}^{\text{tx}}\) 在时长 \(t_{u\to k}\) 内向所属簇 \(k\) 的 CH 送能,其到达 CH 的能量为
\begin{equation}
\begin{aligned}
E_{u\to k}^{\text{MRC}} &= \eta_{u\to k}^{\text{MRC}}\, P_{u\to k}^{\text{tx}}\, t_{u\to k},\\
0 &\le \eta_{u\to k}^{\text{MRC}}\le 1
\end{aligned}
\end{equation}
其中效率 \(\eta_{u\to k}^{\text{MRC}}\) 由耦合系数、品质因数与可能的 RIS 聚焦共同决定(见下)。节点侧驱动损耗由逆变/功放效率 \(\eta_{\text{drv}}\) 计入:\(E_{\text{drv}}=P_{u\to k}^{\text{tx}}t_{u\to k}/\eta_{\text{drv}}\)。
- 电池演化:
\begin{equation}
E_{u}(t+T_f)=\min\{E_{\max},\; E_u(t)+E_{\text{PV}}-E_{\text{use}}-E_{\text{drv}}\}.
\end{equation}

\paragraph{(b) 近场耦合/物理模型}
两线圈谐振耦合效率近似为
\begin{equation}
\begin{aligned}
\eta_{u\to k}^{\text{MRC}} &\approx \frac{k_{u,k}^2 Q_u Q_k}{\big(1+\sqrt{1+k_{u,k}^2 Q_u Q_k}\big)^2},\\
k_{u,k} &= \frac{M_{u,k}}{\sqrt{L_u L_k}}
\end{aligned}
\end{equation}
互感 \(M_{u,k}=M(d,\Delta z,\theta)\) 随水平/垂直偏移与姿态角变化。若存在近场 RIS/超表面,其增益以函数 \(\mathcal{F}_{\text{RIS}}\) 修正:\(\eta\leftarrow \eta\cdot\mathcal{F}_{\text{RIS}}(\text{结构, 频率, 姿态})\)。

\paragraph{(c) 计算与通信负载模型}
- 业务到达:每帧产生数据 \(A_u(t)\)(可泊松/伯努利事件驱动)。
- 通信:低功耗 RF 向 CH 上传,单位比特能耗 \(\epsilon_{\text{tx}}\)、接收能耗 \(\epsilon_{\text{rx}}\)。
- 计算:本地处理量 \(C_u\)(CPU 周期)与动态功耗 \(P\propto f^\gamma\)。

\paragraph{(d) 存储与控制开销}
- 电池容量 \(E_{\max}\)、荷电状态(SOC)与老化可通过库仑计或衰减模型修正。
- 控制开销:接收 Sink/CH 的调度指令,开销合入 \(E_{\text{use}}\)。

\subsubsection{可重构智能反射面(RIS)}
\paragraph{(a) 反射/超表面模型}
RIS 由 \(N\) 个单元组成,等效反射矩阵
\begin{equation}
\begin{aligned}
\boldsymbol{\Theta} &= \operatorname{diag}(\beta_1 e^{j\theta_1},\ldots,\beta_N e^{j\theta_N}),\\
\theta_n &\in \mathcal{Q},\ \ 0\le\beta_n\le 1.
\end{aligned}
\end{equation}
在近场工作区,RIS 通过改变等效边界条件重塑磁场分布,提升 \(k_{u,k}\) 或有效 \(Q\),其对 \(\eta^{\text{MRC}}\) 的作用以经验/仿真查表函数 \(\mathcal{F}_{\text{RIS}}\) 体现。

\paragraph{(b) 硬件与约束}
相位量化比特 \(b\)、更新速率、插入损耗与功耗 \(P_{\text{RIS}}\)(通常极低)。配置比特数计入控制开销。

\subsubsection{RF 簇头(CH)}
\paragraph{(a) 能量模型:仅由簇成员的 MRC 供能}
CH 的能量来源完全来自本簇成员通过 MRC 反向送能:
\begin{equation}
E_{\text{in}}^{\text{MRC}}=\sum_{u\in\mathcal{U}_k} \eta_{u\to k}^{\text{MRC}}\, P_{u\to k}^{\text{tx}}\, t_{u\to k},\quad E_{\text{CH}}(t+T_f)=E_{\text{CH}}(t)+E_{\text{in}}^{\text{MRC}}-E_{\text{use}}^{\text{CH}}.
\end{equation}
其中 CH 的用能包含:
\begin{equation}
E_{\text{use}}^{\text{CH}}=P_{\text{rx}}^{\text{RF}}t_{\text{rx}}+P_{\text{tx}}^{\text{RF}}t_{\text{tx}}+P_{\text{comp}}t_{\text{comp}}+P_{\text{ctrl}}t_{\text{ctrl}}+P_{\text{idle}}t_{\text{idle}}.
\end{equation}
\paragraph{(b) 通信与计算模型}
CH 负责聚合与上行到 Sink 的 RF 数据通信(不向 Sink 传能),链路速率 \(R_{k\to S}\) 与功耗遵循所选制式的能耗模型;本地聚合/压缩计算量 \(C_k\) 由业务决定。
\paragraph{(c) 存储与队列}
维护数据队列 \(Q_k\):\(Q_k(t+1)=\max\{Q_k(t)-\mu_k(t),0\}+\sum_{u}A_u(t)\),其中服务率 \(\mu_k\) 由 CH\(\to\)Sink 的可用时隙与信道条件决定。

\subsubsection{Sink(信息汇聚与调度)}
Sink 负责控制与数据汇聚,不参与能量传输。其功能与模型包括:
- 控制平面:收集能量/业务状态摘要,运行时隙/功率/RIS 配置的调度算法,并下发给 CH/RIS/节点;控制能耗计入各受控节点,不计入能量链路。
- 业务层:汇聚各 CH 上行数据,存储与转发开销可忽略或以固定功耗近似(根据实现)。

% ===================== 场景建模 =====================
\subsection{场景建模}\label{subsec:scene_model}

\subsubsection{全局场景建模:Carrauntoohil 山地数据驱动}\label{sec:carrauntoohil}
我们选取爱尔兰最高峰 Carrauntoohil 山地作为目标场景建模对象。该区域具有典型性:冬季太阳高度角低,山阴面(背阳坡)长时间处于阴影,导致光伏可用性强烈非均匀,这对“节点\(\to\)CH 的 MRC 供能”和“CH\(\to\)Sink 的上行通信”形成耦合约束。

\paragraph{区域选取与数据来源。} 在卫星底图上框选研究区域,由于平台无法直接导出规则网格的高程 DEM,我们采用“稀疏采样 + 插值复原”的方法:
- 在区域内进行 \(N\) 个高程采样(经纬度与地面高程),尽量覆盖山脊、谷地、湖泊与坡向变化显著处;
- 采样点记为 \((\lambda_i,\varphi_i,h_i)\), \(i=1,\ldots,N\)。

\paragraph{坐标与投影。} 以采样中心或首点为原点 \(O\)(经纬高 \(\lambda_0,\varphi_0,h_0\)),将 WGS-84 坐标映射到局部 ENU 米系:
\begin{equation}
\begin{aligned}
x&=(\lambda-\lambda_0)\cos\varphi_0\,R_\oplus,\\
y&=(\varphi-\varphi_0)\,R_\oplus,\\
z&=h-h_0,\quad R_\oplus\approx 6371\,\text{km}.
\end{aligned}
\end{equation}
\begin{figure*}[!t]
  \centering
  \includegraphics[width=.95\linewidth]{sections/figures/elev_coord.png}
  \caption{高程建系与场景边界示意。彩色底图为插值后的高程场 \(H(x,y)\),坐标系为局部 ENU;比例尺与轴向标注对应 \((x,y)\) 的米尺度。该图对应 \(H:\Omega\to\mathbb{R}_{+}\) 的连续近似与研究域 \(\Omega\) 的选择。}
  \label{fig:elev_coord}
\end{figure*}

\paragraph{地形复原:三角网插值 + 栅格化。} 构建地形:
- Delaunay 三角剖分得到 TIN:\(\mathcal{T}=\operatorname{Del}({(x_i,y_i)}_{i=1}^{N})\);
- 在每个三角形内以重心/仿射插值恢复 \(z(x,y)\) 的连续近似(形成分段平面曲面);
- 将 \(z(x,y)\) 以分辨率 \(\Delta\) 栅格化为高度场 \(H[m,n]\),得到规则 DEM(便于快速可见性/阴影运算)。
\begin{figure}[!t]
  \centering
  \includegraphics[width=.95\linewidth]{sections/figures/sampling_tin.png}
  \caption{稀疏采样与描点建模:TIN插值与规则DEM栅格化示意。}
  \label{fig:sampling_tin}
\end{figure}

% —— 新增:采样点分布与三维重建可视化 ——
\begin{figure*}[!t]
  \centering
  \includegraphics[width=.8\linewidth]{sections/figures/points_group.png}
  \caption{采样点的三维分布与分组示意。彩色散点表示不同位置处的高程/分组,可用于说明稀疏采样的空间覆盖与结构特征。}
  \label{fig:points_group}
\end{figure*}

\begin{figure*}[!t]
  \centering
  \includegraphics[width=.8\linewidth]{sections/figures/rebuild_3D.png}
  \caption{基于稀疏样本的地形三维复原结果。通过 TIN/插值获得的连续曲面,展示地形起伏与坡向。}
  \label{fig:rebuild_3D}
\end{figure*}

\paragraph{阴影与辐照修正。} 基于复原 DEM 计算典型日期的太阳位置 \((\alpha(t),\,\gamma(t))\)(高度角/方位角)。对每个栅格中心 \((x,y)\):
- 计算坡度/坡向 \((s,\,\phi)\) 与法向量 \(\mathbf{n}(x,y)\);
- 沿太阳入射方向进行地形遮挡投射,得到遮阴指示 \(S(x,y,t)\in\{0,1\}\);
- 形成光照修正因子 \(\Xi(x,y,t)=S(x,y,t)\cdot\max\{0,\mathbf{n}\cdot\mathbf{s}(t)\}\)。
\begin{figure}[!t]
  \centering
  \includegraphics[width=.95\linewidth]{sections/figures/hillshade_back.png}
  \caption{山阴(背阳坡)与地形自遮挡。给定太阳方向单位向量 \(\mathbf{s}(t)\),对每个 \((x,y)\) 取地形法向 \(\mathbf{n}(x,y)\) 与地平角约束,得到遮阴指示 \(S(x,y,t)\in\{0,1\}\)。若沿 \(\mathbf{s}(t)\) 的射线首次与曲面 \(z=H(x,y)\) 相交于当前点之前,则 \(S=0\)。}
  \label{fig:hillshade_back}
\end{figure}
\begin{figure}[!t]
  \centering
  \includegraphics[width=.95\linewidth]{sections/figures/shadow_model.png}
  \caption{阴影建模与照度修正。瞬时有效入射 \(I(x,y,t)=G_0(t)\,\max\{0,\mathbf{n}(x,y)\!\cdot\!\mathbf{s}(t)\}\,S(x,y,t)\),其中 \(G_0(t)\) 为无地形时的水平面辐照度。将 \(I\) 归一化得到 \(\Xi(x,y,t)=I/G_0\),并用于 PV 获能模型。}
  \label{fig:shadow_model}
\end{figure}
\begin{figure}[!t]
  \centering
  \includegraphics[width=.95\linewidth]{sections/figures/solar_geom_aspect.png}
  \caption{太阳几何与坡向入射差异。上:某一日的太阳高度角随小时变化;下:南坡与北坡(坡度相同、坡向相差约180°)的相对入射(与入射角余弦成正比)理论曲线。冬季高纬度下,南坡全天相对入射显著大于北坡,解释了背阳坡光伏不足。}
  \label{fig:solar_aspect}
\end{figure}

\paragraph{可见性与上行数据链路损耗(仅数据,不传能)。} 在 DEM 上对 CH\(\to\)Sink 的 RF 数据上行进行 LoS 判定与路径损耗参数选择:
- 在两点连线方向对 \(H[m,n]\) 逐步采样,若 \(z_{\text{ray}}(d)\le H(d)\) 则标记为 NLoS;
- LoS/NLoS 决定不同路径损耗指数/阴影项;
- 注:本工作不设 RF 能量源,RIS 仅用于簇内近场磁场聚焦,不参与远场路由。

\paragraph{簇与锚点。} 在湖泊及周边地形语义的引导下布设若干簇与 CH,并在 CH 周边选取一至多处 RIS 候选位以进行近场聚焦对比。Sink 位于山谷低处以模拟背阳通信与汇聚节点场景。图~\ref{fig:anchor_dist} 展示典型锚点间的几何距离,用于“走廊宽度选择”“可见性步长”和“路径损耗自变量”的量化。
\begin{figure*}[!t]
  \centering
  \includegraphics[width=.95\linewidth]{sections/figures/satellite_distance.png}
  \caption{关键锚点与几何距离示意。连线长度标注为 \(\lVert \mathbf{a}-\mathbf{b}\rVert\),用于:(1) 估算走廊半宽 \(W\) 和搜索域 \(\mathcal{C}\);(2) 作为路径损耗与时延的几何自变量;(3) 用于可见性判定的步长约束。}
  \label{fig:anchor_dist}
\end{figure*}

\paragraph{数据产物与可复现性。} 该流程生成:
- 稀疏 TIN 与规则 DEM 栅格(\(H[m,n]\));
- 冬季时段的遮阴时间序列 \(S(x,y,t)\) 与光照修正 \(\Xi(x,y,t)\);
- CH\(\leftrightarrow\)Sink 的 LoS 图与损耗参数场;
- ENU 坐标下的节点/簇/湖泊矢量底图。
上述产物直接驱动仿真与可视化,并与第~\ref{subsec:node_model} 节的 PV 与上行链路模型无缝对接。

\subsubsection{簇内场景建模:近场 MRC + RIS 聚焦}\label{sec:cluster_scene}
\paragraph{几何与错位。} 对簇 \(k\) 内任意节点 \(u\) 与 CH 的几何关系用 \(d_{k,u},\,\Delta z_{k,u},\,\theta_{k,u}\) 表征,互感 \(M_{k,u}=M(d_{k,u},\Delta z_{k,u},\theta_{k,u})\),耦合系数 \(k_{k,u}=M_{k,u}/\sqrt{L_u L_k}\),效率 \(\eta_{k,u}^{\text{MRC}}\) 由前述公式给出。我们将“错位”视作 \(\boldsymbol{\delta}_{k,u}=(\delta x,\delta y,\delta z,\delta\theta)\) 的随机扰动,其分布由部署与环境振动决定。

\paragraph{RIS 对远距与错位的补偿。} 在 CH 周边布置近场 RIS/超表面,作为“能量透镜”提升远距离(较大 \(d_{k,u}\))节点的有效 \(k_{k,u}\) 并降低 \(\eta_{k,u}^{\text{MRC}}\) 对 \(\boldsymbol{\delta}_{k,u}\) 的敏感度。我们以查表/仿真函数 \(\mathcal{F}_{\text{RIS}}(d,\Delta z,\theta;\,\text{结构},\text{频率})\) 对效率进行修正:
\begin{equation}
\eta_{k,u}^{\text{MRC}}\;\leftarrow\;\eta_{k,u}^{\text{MRC}}\cdot \mathcal{F}_{\text{RIS}}(d_{k,u},\Delta z_{k,u},\theta_{k,u}),\quad \partial\eta/\partial\boldsymbol{\delta}\ \text{降低}.
\end{equation}
其中“降低偏导”表示错位鲁棒性的提升。

\paragraph{图示。} 图~\ref{fig:mrc_cluster} 展示了簇内场景:RIS 位于 CH 附近,通过重塑近场磁场,使得远处或偏移/倾斜的节点在 RIS 的聚焦下仍能实现较高的输能效率。该图对应局部层(cluster-level)的建模与设计。
\begin{figure*}[!t]
  \centering
  \includegraphics[width=.95\linewidth]{sections/figures/mrc.png}
  \vspace{-0.4em}
  \caption{簇内场景建模示意:RIS 增强远处节点的 MRC 传能效率,并降低线圈不对齐(平移/倾斜)造成的效率损失。}
  \label{fig:mrc_cluster}
\end{figure*}

\paragraph{调度与约束。} 在帧 \(T_f\) 内,簇内送能时隙集合 \(\{t_{u\to k}\}\) 需满足 \(\sum_{u\in\mathcal{U}_k} t_{u\to k} \le T_f- t_{\text{comm}}-t_{\text{comp}}\)。在 RIS 配置比特与更新速率受限下,选择有限个模式实现对“远距优先/错位补偿”的策略化支持。

\paragraph{小结。} 本节将场景建模拆分为全局(地形、光照、可见性)与簇内(近场几何、RIS 聚焦、调度),并在簇内侧引入基于 RIS 的远距增强与错位鲁棒性建模。
\section{机制设计}\label{sec:mechanism}

\textbf{多RIS协同能量路由系统(能量互联网框架)。} 将RIS面板视为能量路由网络中的主动节点,构建类似数据网络的能量互联网(Energy Routing Network)。系统通过以下机制实现多跳反射链路和能量路径选择:

\textbf{(1) 多RIS协作RF能量路由。} 构建基于RIS的多跳反射能量路径,实现跨山体、跨障碍物的长距离能量传输。具体包括:(i) \textbf{地形感知路由图构建:}集成数字高程模型(DEM)数据,构建包含RIS节点、能量源和簇头的路由图,考虑地形遮挡和阴影效应;(ii) \textbf{多跳路径优化:}采用动态规划、图搜索或基于ADMM的联合相位-路径优化算法,选择最优RIS反射序列,最大化接收功率或平衡多簇能量分配;(iii) \textbf{自适应重配置:}支持RIS面板的动态增删和相位重配置,适应环境变化(如植被生长、部分遮挡)和网络扩展需求。

\textbf{(2) RIS增强MRC近场能量透镜。} 利用可编程超表面重塑近场磁场分布,增强磁耦合效率并提升对线圈错位的鲁棒性。具体包括:(i) \textbf{磁场聚焦设计:}通过HFSS/CST电磁仿真,设计RIS单元几何和周期性,实现近场磁场聚焦;(ii) \textbf{错位补偿:}分析耦合系数$k$、品质因数$Q$和效率$\eta$在错位情况下的变化,设计自适应RIS控制模式补偿空间漂移;(iii) \textbf{动态调整:}根据传感器节点位置不确定性,实时调整RIS相位配置,维持高效能量传输。

\textbf{(3) 跨层能量流协调。} 构建统一的RF--RIS--MRC--WSN能量流图,实现端到端能量优化。具体包括:(i) \textbf{能量流图建模:}将系统建模为多层级能量流图,定义每层的能量约束(功率预算、RIS配置资源、硬件限制);(ii) \textbf{联合优化算法:}设计跨层优化算法,联合分配RF和MRC层之间的功率,选择RIS能量路由,并根据流量负载和剩余能量调度簇级功率分配;(iii) \textbf{信息-能量协同:}通过AOEI机制将能量状态信息新鲜度与能量路由决策耦合,实现信息驱动的能量调度。

\textbf{(4) AOEI优先机制。} 动态设定能量信息年龄阈值,使新鲜度不足的节点优先上报,将信息新鲜度与输能紧迫度耦合,并沿活跃能量路径插入更新。

\textbf{(5) 自适应Lyapunov时长规划(ALDP)。} 基于Lyapunov优化规划每时隙RF多跳与MRC充能时长,稳定AOEI与能量队列,同时最大化接收功率。

\textbf{(6) 能量高效传输机会路由(EETOR)。} 选择多跳RIS路径同时输能并携带ESI,权衡路径损耗、反射增益与AOEI紧迫度,支持RIS面板的增删与重配置。

\textbf{协同关系。} 多RIS协作路由构建能量互联网骨干,RIS增强MRC提供簇内高效充能,跨层协调实现端到端优化,AOEI、ALDP、EETOR三机制绑定信息新鲜度与能量路由,共同提升系统寿命、能量均衡和工程实用性。

\section{实验设计与结果}\label{sec:experiments}
\textbf{场景。} 在山地/遮挡地形中布设多块 RIS 与多个簇:RF 能量源位于向阳坡,簇位于背阴坡;MRC 线圈存在偏移与倾斜。业务与能量负载模拟典型 WSN 采样占空。

\textbf{基线。} (i) 无 RIS 的 RF 输能;(ii) 单 RIS 路由;(iii) 无 RIS 透镜的 MRC;(iv) 不含 AOEI 的 SWIPT 调度;(v) 传统机会路由。

\textbf{指标。} 寿命(首节点失效)、能量均衡(CV)、AOEI 平均/尾部、接收 RF/MRC 功率与端到端效率、控制开销。

\textbf{对比与消融。}
\begin{itemize}[leftmargin=*]
  \item 多 RIS 数量/位置对 RF 路由增益的影响。
  \item RIS 透镜对 MRC 效率与抗偏移性的提升。
  \item AOEI+ALDP+EETOR 联合方案对比逐项消融。
  \item 扩展性:增加簇或 RIS(含移动/UAV RIS)。
\end{itemize}

\textbf{主要结果。} 在总能耗相近的情况下,联合方案的寿命-均衡综合指标较基线提升约 194\%;AOEI 降低信息陈旧度;RIS 透镜改善偏移下的 MRC;多 RIS 在 NLOS 场景显著抬升接收功率。

\section{讨论}\label{sec:discussion}
本节从系统性解读、工程设计启示、复杂度与可部署性、局限与威胁、外推性与未来工作六个维度,对第~\ref{sec:experiments} 节的结果与第~\ref{sec:conclusion} 节的结论进行归纳与反思,力求将“全局几何—局部近场—跨层协调”的三层机制联结为可落地的工程范式。

\paragraph{关键发现的综合解读}
- 全局(G,远场多RIS):在越岭 NLoS 场景,通过受走廊与视域约束的多跳反射路径,CH 侧接收功率与能量可达率显著抬升。双/三跳在复杂地形中呈现“性价比最优”的拐点,过多跳数边际收益递减且控制开销上升。RIS 安装高与走廊半宽存在较稳定的最佳区间(第~\ref{sec:experiments} 节)。
- 局部(L,近场透镜 + MRC):RIS 作为能量透镜在中/大错位与倾斜扰动下显著提升 MRC 效率,等效“有效补能体积”外扩,降低对机械精度与定位误差的敏感度,实现对部署瑕疵的容错(见第~\ref{sec:experiments} 节局部层)。
- 跨层(C,联合调度):在 G+L 的物理增益之上,引入帧级时隙/功率/RIS 模式的联合调度(如 ALDP/EETOR 族)后,能量短缺率进一步下降,SOC 分布更均衡,FND/HND 延长。相位量化在 $b\ge2$ 时已接近饱和,RIS 更新周期可日/小时级少量更新即可维持收益,显示对有限控制预算的鲁棒性。

\paragraph{设计启示与工程建议}
- 资源优先级与预算分配:在固定控制/成本预算下,优先增加 RIS 数量与合理的安装高度,随后再考虑提高相位量化比特;当候选密度受限时,适度放宽走廊半宽以提升几何可连通性。
- 控制面与状态摘要:以“低开销能量状态摘要”驱动跨层决策,将需求感知与多 RIS 能量路由绑定;RIS 配置与调度指令采用分层/增量编码,日级为主、小时级为辅,兼顾收益与开销。
- 近场布局与容错:簇内优先部署少量“主导 RIS 面板”对准 CH—节点典型相对几何,配合 MRC 查表修正以覆盖错位分布的中位与尾部;对超低功耗节点可用离线生成的轻量启发式策略替代在线求解。
- 供能与运维:固定 RIS 建议配套微型光伏或能量驯化模块,移动 RIS(UAV/地面机器人)用于临时补盲与季节性再标定;预留快速现场标定与健康度监测接口以降低长期漂移带来的维护成本。

\paragraph{复杂度、可部署性与实现路径}
- 算法复杂度:跨层优化在帧级引入求解负担,可采用“离线候选裁剪 + 在线轻量决策”的两阶段策略;束搜索宽度与候选规模可按地形复杂度自适应裁剪,以控制在线时延(第~\ref{sec:experiments} 节复杂度指标)。
- 数据与建模依赖:几何连通性强依赖 DEM 质量与遮阴估计精度;建议在部署前进行稀疏测绘与视域验证,并对关键反射段进行小样本现场标定,以提高模型—现实一致性。
- 软硬件协同:RIS 相位控制、MRC 驱动与 WSN MAC 需共享统一的时间参考与极简控制面;建议采用边缘控制器汇聚 RIS 指令并进行状态压缩,减少节点端的协议复杂度。

\paragraph{局限性与威胁}
- 控制与标定开销:RIS 数量增加带来相位标定、插入损耗评估与温漂补偿的额外开销;当环境快速变化(地形积雪、植被季节性变化、临时遮挡)时,需要更快的重配置节律以维持反射链路质量。
- 模型与硬件不完备性:近场耦合与多反射通道存在非理想因素(互耦、量化误差、面板间时延差);DEM 精度与太阳几何估计误差会传导至可达性评估,可能低估极端天气下的波动。
- 安全与鲁棒性:RIS 控制面存在被篡改/干扰的潜在风险,应配套轻量认证与回退策略;跨层调度的误报/漏报会影响能量均衡,需要异常检测与保护带(guard band)。
- 合规与干扰:RF 输能需满足当地辐射与 EIRP 法规,注意与既有通信系统的频谱隔离与旁瓣控制;移动 RIS 应遵循空域/作业安全规范。

\paragraph{适用范围与外推性}
- 场景外推:所提“几何机会 + 近场容错 + 跨层协调”的范式可推广至城市峡谷、风场/林区传感、农田与水域监测等存在 NLoS 约束与错位不确定性的环境;当反射走廊稀缺或 DEM 粗糙时,应优先采用 L+C 以获得稳健收益。
- 技术融合:框架与 SWIPT/波形共设兼容,可将 RIS 相位控制与能量—信息联合度量协同优化;在工业与医疗等边缘场景,能量信息协同的思路同样适用,但需更严格的安全与可靠性保障。

\paragraph{未来工作}
- 学习增强的相位/路由:引入数据驱动的相位码本与多跳路由选择,结合不确定性估计以提升在动态环境下的适应性与样本效率。
- 闭环标定与自监测:构建“测量—校正—验证”闭环,在线估计插入损耗与相位漂移,实现 RIS/MRC 的长期自校准。
- 联合波形与协议共设:探索与 SWIPT/反射调制的协同,面向能量—数据双重目标的帧级协议共设。
- 安全与治理:设计轻量认证、指令完整性保护与异常检测机制,形成可审计的控制面;评估跨域部署的合规路径。
- 机动平台试验:开展 UAV/地面机器人挂载 RIS 的野外原型验证,量化移动性带来的部署灵活性与链路稳健性收益。

综上,本文验证了“以低开销能量状态摘要驱动的跨层优化 + 多 RIS 远场能量路由 + 近场透镜增强 MRC”的统一范式在复杂地形 WSN 的工程可行性与稳健收益;同时也明确了其在控制、标定与安全上的现实边界,为后续走向规模化与长期运行提供了可执行的路线图。
\section{结论}\label{sec:conclusion}

本文面向复杂地形WSN,提出基于RIS的分层混合无线输能架构,构建RIS + WPT + WSN一体化生态系统。本文的核心贡献体现在三个层级:

\textbf{(主)RIS + WPT + WSN一体化生态系统:}通过系统级整合、长期供能、自组织能力和工程实用路线,实现了从理论建模到硬件原型的完整工程实现路径。统一的RF--RIS--MRC--WSN跨层能量流模型协调端到端能量传输,支持大规模WSN部署和未来智能能源基础设施扩展。

\textbf{(辅1)跨山体能量反射:}通过人工电磁路径实现NLOS能量覆盖,特别解决了阴坡补盲问题,突出了山地环境的特殊性与工程意义。RIS构建的可控反射路径突破了地形障碍限制,实现了从向阳坡到背阴坡的能量传输。

\textbf{(辅2)多RIS协同能量路由系统:}将RIS视为能量节点,构建了能量互联网(Energy Routing Network)工程框架。多跳反射链路和能量路径选择机制实现了远距离能量传输和智能能量调度,为未来能量网络奠定了基础。

通过跨层优化(如 ALDP、EETOR 等),系统将需求感知与 RIS 辅助的 RF 多跳路由、RIS 增强的 MRC 充能紧密耦合,在不增加能耗的前提下显著提升寿命与能量均衡。实验显示相较传统 RF/MRC 基线及消融方案均有显著增益。

未来工作:加强RIS控制与数据安全,探索学习驱动的相位优化,扩展能量互联网框架到更多应用场景,并在移动RIS(如无人机载RIS)实地试验中验证系统的工程实用性。


% ========== 参考文献 ==========
\begin{thebibliography}{00}
\bibitem{ref:aoei} A. Author, B. Author, and C. Author, ``Age-of-energy-information scheduling for wireless sensor networks,'' \emph{IEEE Trans. Wireless Commun.}, vol. 00, no. 0, pp. 1--12, 2025.
\bibitem{ref:eetor} F. Author \emph{et al.}, ``Opportunistic routing for simultaneous information and energy transfer,'' in \emph{Proc. IEEE INFOCOM}, 2025, pp. 1--6.
\end{thebibliography}


% ========== Biographies ==========
\begin{IEEEbiographynophoto}{作者姓名}
作者简介。
\end{IEEEbiographynophoto}

\begin{IEEEbiographynophoto}{合作者姓名}
作者简介。
\end{IEEEbiographynophoto}

\end{document}

